\section{Introduction}

Bitcoin \cite{bitcoin} is the first decentralized currency. Several
cryptocurrencies have surfaced since, many of which use different consensus
rules. Such different proposals for consensus rules must necessarily be
implemented either as an altcoin or a hard-fork. Pegged sidechains provide an
alternative: While consensus rules can be different across blockchains, it is
useful to have one unit of value which can be moved between various
blockchains. This allows for consensus rules innovation and experimentation.

In the simplest version of sidechains, the nodes participating in the consensus
protocol must all maintain the different blockchains. This is not efficient and
adoption will be hindered. The solution for this is to provide succinct
stand-alone proofs that the winning chain of a block tree is a particular one.
These succinct proofs prove that proof-of-work took place without presenting
the actual proof-of-work. Succinctness requires that these proofs are
significantly shorter than presenting the whole chain. Technically, if the
claimed winning blockchain has a length of $|\chain|$ blocks, we want the
succinct proofs to have a size of $O(polylg|\chain|)$. These short proofs can
then be subsequently included in sidechain blocks where funds are locked in one
blockchain so that they can become available in another. This is in vast
improvement as compared to SPV proofs which are $O(|\chain|)$.

Proofs-of-proofs-of-work were explored in KLS (TODO: reference). These proofs
had the disadvantage of requiring interaction between a prover and a verifier,
making them unsuitable for sidechain adoption.

We extend these proofs to make them non-interactive, making them suitable for
sidechain use. We generalize the idea of proof-of-mining into not just being
able to talk about what the current winning chain suffix is, but allowing
provers and verifiers to prove more generic predicates. Such predicates can
express simple ideas, such as the inclusion of a transaction in the longest
chain, to complex ideas such as the modification of an account's balance over
time.

We provide generic definitions for what a PoPoW system should achieve. Next,
the theoretical construction for succinct non-interactive
proofs-of-proofs-of-work is presented. Subsequently, we prove the construction
is succinct and secure. Next, we sketch a mechanism by which bitcoin can be
upgraded to adopt our scheme without having to orchestrate a hard or a soft
fork. Finally, we evaluate the feasibility of our construction by proposing
specific values for the security parameters and analyzing the size of our
actual proofs in the real bitcoin blockchain.
