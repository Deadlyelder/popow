\section{Introduction}
Cryptocurrencies such as Bitcoin~\cite{bitcoin} and Ethereum~\cite{ethereum} have seen tremendous success in recent years.
These are peer-to-peer networks that maintain a globally consistent transaction ledger, using a new consensus protocol based on proof-of-work (PoW) puzzles. Worker nodes called ``miners'' expend computational work in order to reach agreement on the state of the network.
Clients on the network, such as mobile phone apps, must verify these proofs-of-work in order to determine the correct view of the network's state, as necessary to transmit and receive payments.

In this work we introduce, analyze and implement a new primitive,
Noninteractive Proofs of Proof-of-Work (NIPoPoWs), which can be adapted into existing cryptocurrencies to support more efficient clients.
Unlike a traditional blockchain client, which must verify the entire chain of proofs-of-work, which grows linearly over time, clients based on NIPoPoWs require resources only logarithmic in the length of the blockchain.

There are three trends that motivate the need for more efficient clients: specifically clients that can verify transactions without having to process the whole blockchain.
First, while Bitcoin remains by far the largest and most widely used cryptocurrency, the ecosystem has become significantly more diverse. Today there are hundreds of competitors (commonly called ``altcoins''), and Bitcoin's dominance in the market is at an all time low.~\cite{marketcap}.\footnote{There are numerous problems with using the market cap as a measure of economic size, such as defining what it means to quantify the quantity ``in circulation'' counting the number of this is outside the scope of our paper.}
  We envision that our protocol will form the basis of an efficient multi-blockchain client, which could efficiently support payments using hundreds of separate cryptocurrencies.

  Second, cryptocurrencies are increasingly used as components of larger systems (the cryptocurrency system is typically called a ``blockchain'' when its currency feature is not salient).
  As one example, a recent system called Catena~\cite{catena} uses the Bitcoin blockchain as an authenticated log; the browser embeds a Bitcoin client that uses the blockchain to validate HTTPS certificates.
  The bandwidth costs that are tolerable for a dedicated Bitcoin wallet are likely unacceptable in such embedded contexts. Our techniques can directly reduce the cost of Catena and similar systems.

  Finally, there has been significant interest in the development of ``cross-chain'' applications, i.e. logical transactions that span multiple separate cryptocurrencies.
  Simple cross-chain transactions are feasible today: the most well-known is the atomic exchange,~\cite{tiernolan} e.g. a trade of Bitcoins for Ether.
  However, more sophisticated applications could be enabled by our protocol, which would allow one cryptocurrency to embed a client of a separate cryptocurrency.
  This concept initially popularized by a proposal by Back et al.
  ~\cite{sidechains} particularly can be used to avoid a difficult upgrade
  process: a new blockchain with additional features, such as experimental
  opcodes, can be backed by deposits in the original Bitcoin currency, obviating the need to bootstrap new currencies from scratch.
  This has been heralded as a key enabling scalability of the Bitcoin ecosystem;
  however, the sidechains proposal is incomplete, as it does not provide a protocol that satisfies a natural security definition (e.g., where attacks succeed with only negligible probability). Our work is the first to answer the open problem posed by Back et al.

\noindent
\paragraph{Our contributions.}
In summary,  we make the following contributions in this paper.

%  \begin{itemize}[leftmargin=3.2mm]
%\item
  Our main technical contribution is the introduction of
    a new cryptographic
     primitive called Non-Interactive Proofs of Proof of Work (NIPoPoW). We present a formal model and a provably secure
     instantiation of NIPoPoWs in the model of \cite{backbone}.
     Our contribution builds on previous work by  Kiayias et al., who posed
     (interactive) Proofs-of-Proof-of-Work (PoPoWs) \cite{KLS}.
     Our definition fills the gap in terms of security modeling and design that
      existed in previous proposals, especially the notion of cumulative
      ``Dynamic Member Multisignature'' by Back et al., \cite{}. Moreover
      it is also the first truly non-interactive PoPoW since even the
      efficient PoPoW construction of \cite{KLS} is susceptible to forced
      interactions when an adversarial prover is present.

We provide concrete parameterization and empirical analysis focusing to show the potential savings of our approach versus existing clients.
Using actual data from the bitcoin blockchain we quantify the actual savings of NIPoPoWs over
the previous known techniques of constructing efficient SPV verifiers.

We describe two safely deployable upgrade paths for NIPoPoWs that existing cryptocurrencies can  adopt to support our efficient clients. Specifically
   we first outline how a ``soft-fork''  is feasible for NIPoPoWs so that the
   adoption of the update can be decided by the underlying PoW based consensus mechanism. We then demonstrate that an even less disruptive update mechanism is also feasible that we term a ``velvet fork'' and which may be of independent interest. In a velvet fork update the clients that have forked from the original implementation continue to be fully compatible with unupdated clients. It follows that in a velvet fork, the blockchain system can remain supported by a forked software codebase indefinitely, while it can still enjoy, at least in proportion, some of the (efficiency in our case) benefits of the update without any of the  security downsides.
%  \end{itemize}

%In the simplest version of sidechains, the nodes participating in the consensus
%protocol must all maintain the different blockchains.
%This is not efficient and
  %adoption will be hindered.
  \ignore{
The solution for this is to provide succinct
stand-alone proofs that the winning chain of a block tree is a particular one.
These succinct proofs prove that proof-of-work took place without presenting
the actual proof-of-work. Succinctness requires that these proofs are
significantly shorter than presenting the whole chain.

Technically, if the
claimed winning blockchain has a length of $|\chain|$ blocks, we want the
succinct proofs to have a size of $O(polylg|\chain|)$. These short proofs can
then be subsequently included in sidechain blocks where funds are locked in one
blockchain so that they can become available in another. This is in vast
improvement as compared to SPV proofs which are $O(|\chain|)$.

Proofs-of-proof-of-work were explored in \cite{KLS}. These proofs
had the disadvantage of requiring interaction between a prover and a verifier,
making them unsuitable for sidechain adoption.

We extend these proofs to make them non-interactive, making them suitable for
sidechain use. We generalize the idea of proof-of-mining into not just being
able to talk about what the current winning chain suffix is, but allowing
provers and verifiers to prove more generic predicates. Such predicates can
express simple ideas, such as the inclusion of a transaction in the longest
chain, to complex ideas such as the modification of an account's balance over
time.

We provide generic definitions for what a PoPoW system should achieve. Next,
the theoretical construction for succinct non-interactive
proofs-of-proof-of-work is presented. Subsequently, we prove the construction
is succinct and secure. Next, we sketch a mechanism by which bitcoin can be
upgraded to adopt our scheme without having to orchestrate a hard or a soft
fork. Finally, we evaluate the feasibility of our construction by proposing
specific values for the security parameters and analyzing the size of our
actual proofs in the real bitcoin blockchain.
}
