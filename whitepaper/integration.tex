\section{Implementation and Parametrization}
In this section

\subsection{Prototype Implementation}
We provide an implementation of our protocol in Python. Our protocol consists of a prover, 
Our simulation evalute the

\subsection{Concrete Parameterization}
In this section we work through an example of choosing concreter parameters for our scheme.

We make two assumptions.

First, as in all Backbone-based security models, we must make an assumption that a majority of the network's overall hashpower is honest. Letting $\alpha$ represent an assumed bound on the strength of an adversary,
we require at most $\alpha < \frac{1}{2}$.
We take $\alpha = 20\%$.

We must make an assumption about the strength (in hashpower) that we can tolerate.

\begin{figure}
  \caption{Cost for varying functions of.}
\end{figure}
\anote{Include a graph here!}


\subsection{Comparison with alternatives.}

\paragraph{.}

\begin{table}
  \caption{Comparison of Sidechains Techniques ($B$ blocks in between transactions)}
  \begin{tabular}{llll}
    {\bf Technique}  & {\bf Cost}       & {\bf Rounds}    & {\bf Trust Model} \\
    Our technique    & $O(m polylog B)$ &              & \\
    PoPoW~\cite{kls} & $O(m polylog B)$ &              & \\
    Plain SPV        &   $O(B)$         & & \\
Drivechains~\cite{sztorc}& $O(1)$       &              & \\
    Federated        &     $O(1)$       &              & Centralized \\
  \end{tabular}
\end{table}

In the case of 

% We have identified flaws in the proof of the original KLS paper?

\section{Integration Scenarios}

In this section we discussion several integration scenarios that can benefit.

\subsection{Upgrade Paths}

\paragraph{Proof of Work.}

\paragraph{Proof of Stake.}

\subsection{Application Scenarios}

\paragraph{Multi-blockchain wallets.}
As mentioned, one application of our technique is the potential for a multi-cryptocurrency client.

At the current time of writing, a total of 731 cryptocurrencies are listed on 
\url{https://coinmarketcap.com/}.

\anote{graph of market caps of proof-of-work based cryptocurrencies}

Multiple online services provide access to multiple cryptocurrencies. These come with the drawback of relying on a third party.
Similarly, many multi-asset wallet applications, such as Jaxx\footnote{\url{https://jaxx.io/}} and Coinomi\footnote{\url{https://coinomi.com/}} do not connect to the peer-to-peer networks.

\begin{table}
  \caption{Cost of header chains for all active Proof-of-Work-based cryptocurrencies (collected from \url{Coinwarz.com})}
\small
  \begin{tabular}{l|ll|ll}
    {\bf Hash} & {\bf \# Instances} & {\bf Growth rate} & {\bf Size today} &  \\\\
    Scrypt  & 33  & 4.2 GB  & 4.0 MB / day \  \\
    SHA-256  & 19  & 2.5 GB  & 1.6 MB / day \  \\
    X11  & 5  & 591.2 MB  & 556.3 kB / day \  \\
    Quark  & 3  & 647.9 MB  & 518.4 kB / day \  \\
    NeoScrypt  & 3  & 420.7 MB  & 230.4 kB / day \  \\
    CryptoNight  & 2  & 199.0 MB  & 115.2 kB / day \  \\
    EtHash  & 2  & 588.6 MB  & 921.6 kB / day \  \\
    Groestl  & 2  & 300.3 MB  & 184.2 kB / day \  \\
    Equihash  & 2  & 17.7 MB  & 92.2 kB / day \  \\
    Keccak  & 1  & 161.1 MB  & 115.2 kB / day \  \\
    Blake-256  & 1  & 45.8 MB  & 38.4 kB / day \  \\
    Lyra2REv2  & 1  & 57.4 MB  & 46.1 kB / day \  \\
    \hline
    Total  & 74   &  9.7 GB  & 8.4 MB  / day  \\
  \end{tabular}
\end{table}

In Table 1, we show the costs associated with running a client supporting multiple cryptocurrencies. Our estimates are based on an assumption that the header for each block is 80 bytes, as is the case with Bitcoin. For EtHash based currencies, this is an underestimate, since according to the Ethereum whitepaper,\footnote{\url{https://github.com/ethereum/wiki/wiki/White-Paper}}
the header of an Ethereum block is approximately 200 bytes.


Two obstacles. First, require making multiple connections to the peer-to-peer network.


\paragraph{Catena.}

In Catena, propose development of.

Develop a header relay network as a way of releasing.

\subsection{Sidechains}




\section{Related Work}

\paragraph{Cross Chain Transactions.}
An application of our technique is for 


\paragraph{BTCRelay.}
BTCRelay~\cite{btcrelay} is an Ethereum smart contract written in Ethereum that acts as an SPV client for the Bitcoin blockchain.
This allows an Ethereum smart contract to act .

Users can interact with the client to submit itcoin

At the current time.
