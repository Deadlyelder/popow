\subsection{Security}

We will call a query to the random oracle $\mu$-\textit{successful} if the
random oracle returns a value $h$ such that $h \leq 2^{-\mu}T$.

In order to prove our construction secure, we first define the boolean random
variables $X_r^\mu$, $Y_r^\mu$ and $Z_r^\mu$, akin to the random variables
$X_r$, $Y_r$ and $Z_r$ of the backbone setting\cite{backbone}. For each round
$r$ and for each query index $j$ and for each $k$ adversarial party index (out
of $t$), define these random variables as follows.  If at round $i$ an honest
party obtains a PoW with $id < 2^{-\mu}T$, then $X_r^\mu = 1$, otherwise
$X_r^\mu = 0$. If at round $r$ exactly one honest party obtains a PoW with $id <
2^{-\mu}T$, then $Y_r^\mu = 1$, otherwise $Y_r^\mu = 0$. Regarding the
adversary, if at round $r$ the $j$-th query of the $k$-th corrupted party is
$\mu$-successful, then $Z^\mu_{ijk} = 1$, otherwise $Z^\mu_{ijk} = 0$. Further
define $Z^\mu_r = \sum_{k=1}^t \sum_{j=1}^q Z^\mu_{ijk}$, where $q$ denotes the
total number of queries. For a set of rounds $S$, let $X^\mu(S) = \sum_{r \in S}
X_r$ and similarly define $Y^\mu(S)$ and $Z^\mu(S)$.

We now extend the definition of a \textit{typical execution} from the backbone
setting\cite{backbone}.

\begin{definition}{(Typical execution)}
    An execution of the protocol is $(\epsilon, \eta)$-\textit{typical} if:

    \textnormal{\bf Block counts don't deviate.}
    For all $\mu \geq 0$ it holds that for any set $S$ of consecutive rounds
    with $|S| \geq 2^\mu \eta\kappa$, the following hold:

    \begin{itemize}
        \item $(1 - \epsilon)E[X^\mu(S)] < X^\mu(S) < (1 + \epsilon)E[X^\mu(S)]$ and $(1 - \epsilon)E[Y^\mu(S)] < Y^\mu(s)$.
        \item $Z^\mu(S) < (1 + \epsilon)E[Z^\mu(S)]$.
    \end{itemize}

    \textnormal{\bf Round count doesn't deviate.}
    Let $S$ be a set of consecutive rounds such that $Z^\mu(S)
    \geq k$ for some security parameter $k$. Then we have that $|S| \geq (1 -
    \epsilon)2^\mu\frac{k}{pqt}$ with overwhelming probability in $k$.

    \textnormal{\bf Chain regularity.}
    No insertions, no copies, and no predictions \cite{backbone} have occurred.
\end{definition}

\begin{theorem}[Typicality]\label{thm.typicality}
Executions are $(\epsilon, \eta)$-typical with overwhelming
probability in $\kappa$.
\end{theorem}
\begin{proof}
    % \dznote{Copy proof sketch for block counts and regularity}

    \textbf{Block counts and regularity. }
    For the blocks count and regularity, we refer the reader to \cite{backbone}
    for the full proof.

    \textbf{Round count. }
    First, observe that $Z_{ijk}^\mu \sim \textsf{Bern}(2^{-\mu}p)$ and these
    are jointly independent. Therefore $Z_S^\mu \sim \textsf{Bin}(tq|S|,
    2^{-\mu}p)$ and $|S| \sim \textsf{NB}(Z_S, 2^{-\mu}p)$. So $\mathbb{E}(|S|) =
    2^\mu\frac{Z_S}{pqt}$. Applying a tail bound to the negative binomial
    distribution, we obtain that $\Pr[|S| < (1 - \epsilon)\mathbb{E}(|S|)] \in
    \Omega(\epsilon^2 m)$. \Qed
\end{proof}

The following lemma is at the heart of the security proof that will follow.

% \aknote{It might be better if instead of $\chain'_\mathcal{A}$
% you just count the number of $\mu_\mathcal{A}$ blocks the adversary produced.}

\begin{lemma}\label{lem.level-comparison}
    Suppose $S$ is a set of consecutive rounds $r_1 \ldots r_2$
    and $\chain_B$ is
    a chain adopted by an honest party at round $r_2$ of a typical execution.
    Let $\chain_B^S = \{b \in
    \chain_B: b \textnormal{ was generated during } S\}$. Let $\mu_\mathcal{A},
    \mu_B \in \mathbb{N}$. Suppose $\chain_B^S\upchain^{\mu_B}$ is good.
    Suppose $\chain'_\mathcal{A}$ is a $\mu_\mathcal{A}$-superchain containing
    only adversarially generated blocks generated during $S$ and suppose that
    $|\chain'_\mathcal{A}| \geq k$.
    Then:

    \begin{equation*}
    2^{\mu_\mathcal{A}}|\chain'_\mathcal{A}| < \frac{1}{3}2^{\mu_B}|\chain_B^S\upchain^{\mu_B}|
    \end{equation*}
\end{lemma}
\begin{proof}
    From $|\chain'_\mathcal{A}| \geq k$ we know that $Z_S \geq k$. Applying
    Theorem~\ref{thm.typicality}, we conclude that $|S| \geq (1 -
    \epsilon')2^{\mu_A}\frac{1}{pqt}|\chain'_\mathcal{A}|$.

    Applying the chain growth theorem \cite{backbone} we obtain that $|\chain_B^S|
    \geq (1 - \epsilon)f|S|$. But from the goodness of $\chain_B^S\upchain^{\mu_B}$, we
    know that $|\chain_B^S\upchain^{\mu_B}| \geq (1 -
    \delta)2^{-\mu_B}|\chain_B^S|$. Therefore $|\chain_B^S\upchain^{\mu_B}| \geq
    2^{-\mu_B}(1 - \delta)(1 - \epsilon)f(1 -
    \epsilon')2^{\mu_\mathcal{A}}\frac{1}{pqt}|\chain'_\mathcal{A}|$ and so:

    \begin{equation*}
    2^{\mu_\mathcal{A}}|\chain'_\mathcal{A}|
    <
    \frac{pqt}{(1 - \delta)(1 -
    \epsilon')(1 - \epsilon)f}2^{\mu_B}|\chain_B^S\upchain^{\mu_B}|
    \end{equation*}
    \Qed
\end{proof}

\begin{definition}[Adequate level of honest proof]
Let $\pi$ be an honestly generated proof constructed upon some adopted chain
$\chain$ and let $b \in \pi$.

Then $\mu'$ is defined as follows:

\begin{equation*}
\mu' = \max\{\mu: |\pi\{b:\}\upchain^\mu| \geq \max(m + 1, (1 -
\delta)2^{-\mu}|\pi\{b:\}\upchain^\mu\downchain|)\}
\end{equation*}

We call $\mu'$ the \textit{adequate} level of proof $\pi$ with respect to
block $b$ with security parameters $\delta$ and $m$. Note that the adequate
level of a proof is a function of both the proof $\pi$ and the chosen block $b$.
\end{definition}

\begin{lemma}\label{lem.allblocks}
Let $\pi$ be some honest proof generated with security parameters $\delta, m$.
Let $\chain$ be the underlying chain, $b \in \chain$ be any block and $\mu'$ be
the adequate level of the proof with respect to $b$ and the same security
parameters.

Then
\begin{equation*}
\chain\{b:\}\upchain^{\mu'} = \pi\{b:\}\upchain^{\mu'}
\end{equation*}
\end{lemma}
\begin{proof}
    $\pi\{b:\}\upchain^{\mu'} \subseteq \chain\{b:\}\upchain^{\mu'}$ is trivial.

    For the converse, we will show that for all $\mu^* > \mu'$, we have that in
    the iteration of the Prove \texttt{for} loop with $\mu = \mu^*$, the block
    stored in variable $B$ preceeds block $b$ in $\chain$.

    Suppose $\mu = \mu^*$ is the first \texttt{for} loop iteration during which the
    property is violated. Clearly this cannot be the first iteration, as there
    $B = \chain[0]$ and Genesis preceeds all blocks, including itself. By the
    induction hypothesis we see that during the iteration $\mu = \mu^* + 1$,
    block $B$ preceeded block $b$. But from the definition of $\mu'$ we know
    that $\mu'$ is the highest level for which
    $|\pi\{b:\}\upchain^{\mu'}[1:]|
    \geq \max(m, (1 -
    \delta)2^{-\mu'}|\pi\{b:\}\upchain^{\mu'}[1:]\downchain|)$.

    Hence, this
    property cannot hold for $\mu^* > \mu'$ and therefore
    $|\pi_B\{b:\}\upchain^{\mu_B^*}[1:]| < m$ or $\lnot
    \textsf{local-good}_\delta(\pi\{b:\}\upchain{\mu^*}[1:], \chain, \mu^*)$.

    In case \textsf{local-good} is violated, variable $B$ remains unmodified and
    the induction step holds. If \textsf{local-good} is not violated, then
    $|\pi\{b:\}\upchain^{\mu^*}[1:]| < m$ and so $\pi\upchain^{\mu^*}[-m]$
    preceeds $b$, and so we are done.
    \Qed
\end{proof}


% \aknote{maybe we need a definition of the ``compared level'' }

\begin{lemma}
Suppose the verifier evaluates $\pi_\mathcal{A} \geq \pi_B$ in a protocol
interaction where $B$ is honest and assume during the comparison that the
compared level of the honest party is $\mu_B$. Let $b =
\textsf{LCA}(\pi_\mathcal{A}, \pi_B)$ and let $\mu_B'$ be the adequate
level of $\pi_B$ with respect to $b$. Then $\mu_B' \geq \mu_B$.
\end{lemma}
\begin{proof}
    Because $\mu_B$ is the compared level of the honest party we have:

    $2^{\mu_B}|\chain\{b:\}\upchain^{\mu_B}| > 2^{\mu_B}|\chain\{b:\}|$

    We will prove the claim by contradiction. Suppose $\mu_B' < \mu_B$. Because
    by definition $\mu_B'$ is the maximum level such that
    $|\pi_B\{b:\}\upchain^\mu[1:]| \geq \max(m, (1 -
    \delta)2^{-\mu}|\pi_B\{b:\}\upchain^\mu[1:]\downchain|)$, therefore
    $\mu_B$ does not satisfy this condition. But we know that
    $|\pi_B\{b:\}\upchain^{\mu_B}[1:]| \geq m$ because $\mu_B$ was selected by
    the Verifier. So therefore,

    \begin{equation*}
    2^{\mu_B}|\chain_B\{b:\}\upchain^{\mu_B}| < (1 - \delta)|\chain\{b:\}|
    \end{equation*}

    By the fact that $\mu_B'$ satisfies goodness, we have that:

    \begin{equation*}
    2^{\mu_B'}|\chain_B\{b:\}\upchain^{\mu_B'} > (1 - \delta)|\chain\{b:\}|
    \end{equation*}

    From the last two equations, we obtain that:

    \begin{equation*}
    (1 - \delta)|\chain\{b:\}| > 2^{\mu_B'}|\chain\{b:\}\upchain^{\mu_B'}|
    \end{equation*}

    But the last two equations contradict, hence the claim holds.
    \Qed
\end{proof}

\begin{restatable}{theorem}{restateThmSecurity}
    \label{thm.security}
    The non-interactive proofs-of-proof-of-work construction for $k$-stable
    suffix-sensitive predicates is secure with overwhelming probability in
    $\kappa$.
\end{restatable}

\ifonecolumn
\import{./}{proofs/security.tex}
\else
Full proofs are included in Appendix~\ref{sec.proofs}.
\fi

\begin{remark}[Variance attacks]
    \label{rmk.variance}
    The critical issue addressed by this security proof is to avoid Bahack-style
    attack \cite{bahack} where the adversary constructs ``lucky'' high-difficulty
    superblocks without filling in the underlying proof-of-work in the lower
    levels. Observe that, while setting $m = 1$ ``preserves'' the proof-of-work in
    the sense that expectations remain the same, the probability of an adversarial
    attack becomes approximately proportional to the adversary power if the
    adversary follows a suitable strategy (for a description of such a strategy,
    see the parametrization section). With higher values of $m$, the probability of
    an adversarial attack drops exponentially, even though they maintain constant
    computational power, and hence satisfy a strong notion of security.
\end{remark}
