\section{Security}

\begin{theorem}
    The non-interactive proofs-of-proofs-of-work construction is secure.
\end{theorem}

\begin{proof}
    The critical issue is to avoid Bahack-style attack \cite{bahack} where the
    adversary constructs "lucky" high-difficulty superblocks without filling in
    the underlying proof-of-work in the lower levels.
\end{proof}

\begin{itemize}
    \item
        We will enforce both a lower bound and an upper bound for the length of
        each superchain in the proof. The lower bound will be $m$. The upper
        bound could be something like $5m$ or $m^2$? We want a limit such that
        the probability of the proof requiring this size for an honest party is
        negligible, therefore larger proofs can be automatically dismissed as
        adversarial. That way we can ensure that the proofs will necessarily be
        sublinear in size, but can also help with the security proof.
    \item
        When comparing an adversarial and an honest proof, they can be at the
        same level or at different levels.
    \item
        If they are at the same level, then the lowest common ancestor of the
        two chains will necessarily be close to the end of the proof, except
        with negligible probability. This stems from the fact that in the
        opposite case, the adversary would have to produce proof-of-work faster
        than the honest parties and for a significant amount of time.
    \item
        Therefore, if the proofs are at the same level, they can be compared by
        recursing into a lower level, one level at a time. The negligibility
        argument for the lowest common ancestor being near the suffix applies
        to all levels, and hence recursion should be possible.
    \item
        If the adversarial proof is of a higher superblock level than the
        honest proof, then this proof will necessarily extend some honest block
        included in the honest proof. If the honest proof is of level $\mu$ and
        the adversarial proof is of level $\mu + 1$ this means that the
        adversary was able to produce one or more $(\mu + 1)$-level blocks
        which extended the $(\mu + 1)$-level blocks that exist in the
        $\mu$-level proof of the honest parties.
    \item
        By the previous reasoning, the adversary should not be able to produce
        too many of these independently of the honest party, and therefore they
        will share a lowest common ancestor near the suffix. Proofs comparison
        can therefore continue recursively.
    \item
        It seems unlikely that the adversary will be able to present a proof of
        a higher level than the honest parties for $i >> 1$ or even $i > 1$.
        Perhaps we can think of a formal argument for this.
    \item
        In case the adversarial proof is of a lower level than the honest
        proof, it still cannot exceed the enforced upper bound of proof size.
        Assume the upper bound were $2m$. That would necessarily entail that
        the honest party has more proof of work than the adversarial party.
        Does the same argument work for other upper bounds?
    \item
        In the above, proof-of-work amount comparison must cross different
        superchains. i.e. the total amount of proof of work considered in each
        case must be a weighted sum of proof-of-work for each level.
    \item
        This weighted sum could look as follows:
        \begin{equation}
            PoW =
            \sum_{j = \mu}^{\mu + i}
            {|\overline{\Pi}_j \setminus \overline{\Pi}_{j - 1}|}
            + \overline{\Pi}_{\mu - 1}
        \end{equation}
    \item
        When the adversary presents a lower-level superchain of level $\mu$:
        The honest party has already presented a lower-level superchain of at
        least $m$ blocks. The adversarial lower-level superchain can share some
        block of these at level $\mu$ (note that the honest party has not
        provided $\mu$-level superblocks all the way back to genesis, so a
        common ancestor does not trivially exist). If they share a lowest
        common ancestor within the $m$ blocks of level $\mu$ presented by the
        honest party, then the proof comparison can continue as usual.
    \item
        On the other hand, if there are no shared blocks between the
        adversarial proof and the $\mu$-level superchain of the honest proof,
        then that means that the adversary has either produced their own
        $\mu$-level blocks or has reused earlier $\mu$-level honest blocks.
    \item
        Reusing $\mu$-level honest blocks cannot be done arbitrarily. Once
        enough blocks have been used, the proof will be upgraded to $\mu +
        1$-level.
    \item
        Adversarially producing $\mu$-level blocks is possible, but $\mu +
        1$-level blocks generated during this adversarial mining procedure must
        be discarded after $m - 1$ blocks of superblock level $\mu + 1$ have
        been generated.
    \item
        The argument then goes as follows: Assuming honest superblock proof
        level $\mu + i$, it is hard for the adversary to produce $m$ blocks of
        level $\mu + i - 1$, while the honest proof already has $m$ blocks of
        that level. It becomes easier for the adversary as we approach level
        $\mu$. But proof-of-work has to be added across multiple levels to
        achieve a fair comparison. I'm not sure how to do that exactly.
    \item
        Note than in practice succinct proofs are just sets of blocks. The
        proof level is therefore extracted by the validator by comparing
        the blockids against $T / 2^i$.
\end{itemize}
