\section{An attack against proofs of proof-of-work}

In this
section, we revisit the construction for interactive proofs of proof-of-work from \cite{KLS} and its security.  Understanding their construction will be helpful in
eventually constructing non-interactive proofs of proof-of-work that are secure.
Then, we show that a powerful attacker can break chain superquality with
non-negligible probability and we construct a concrete attack based on it.
Finally, we show that their protocol is actually insecure by
constructing a concrete attack against their scheme that allows an adversary
to double spend with non-negligible probability.

\subsection{The interactive proofs of proof-of-work protocol}

\dznote{Include shortened pseudocode for KLS construction}

The interactive proof of
proof-of-work of \cite{KLS} operates as follows.
The core of the verification process is distinguishing between
two candidate proofs $(\pi_A, \chi_A)$  and $(\pi_B, \chi_B)$.
In case  $\pi_A  = \pi_B$, the decision can be drawn immediately (without interaction) based
on $\chi_A,\chi_B$.
Otherwise, the verifier queries the two provers for their claimed
anchored superchains $\pi_A\upchain^\mu$ and $\pi_B\upchain^\mu$ at a
certain level $\mu$. The verifier starts querying at the highest possible level
$\mu$ and successively descends until level $\mu$ is sufficiently low such that
the lowest common ancestor block $b = LCA(\pi_A\upchain^\mu,
\pi_B\upchain^\mu)$ of the two provers' $\mu$-superchains is $m$ blocks deep
for at least one of the provers. More specifically, the querying stops at such
$\mu$ when $max(|\pi_A\upchain^\mu\{b:\}|, |\pi_B\upchain^\mu\{b:\}|) \geq
m$ holds. Subsequently, the winner is designated as the prover with the most
blocks after $b$ at that level. More precisely, the winner is $A$ if
$|\pi_A\upchain^\mu\{b:\}| \geq |\pi_B\upchain^\mu\{b:\}|)$ and otherwise it
is $B$. The communication overhead is  reduced by only requesting
blocks after the purported LCA.

\subsection{Attacking chain superquality}
\label{subsec.superquality-attack}

We present next a general attack against chain superquality
that works against any blockchain protocol that incorporates
the interlink data structure, cf. Section~\ref{sec.interlink}.
Observe that in expectation, in a
completely honest setting with random network scheduling, the honestly adopted
chain will be $\mu$-locally-good for sufficiently long subchains everywhere as
proven in Lemma~\ref{lem.localgood}.

Let us now construct an adversary to break this property at level $\mu$. Suppose
the adversary $\mathcal{A}$ controls a portion of the hashing power equal to
$t/n$. $\mathcal{A}$ then works as follows. Assume she wants to attack the
honest party $B$ in order to have him adopt a chain $\chain_B$ which has a bad
distribution of superblocks, i.e. such that local goodness is violated in some
sufficiently long subchain. She continuously determines the current chain
$\chain_B$ adopted by $B$ (as honest chain adoption is deterministic, this can
be easily done by inspecting the messages exchanged by the honest parties). The
adversary starts playing after $|\chain_B| \geq 2$. If
$\textit{level}(\chain_B[-1]) < \mu$, then $\mathcal{A}$ remains idle. However,
if $\textit{level}(\chain_B[-1]) \geq \mu$, then $\mathcal{A}$ attempts to mine
an adversarial block $b$ on top of $\chain_B[-2]$. If she is successful, she
broadcasts $b$. The adversarial mining continues until party $B$ adopts a new
chain, which can be either due to two reasons. Either the adversary managed to
successfully mine $b$ on top of $\chain_B[-2]$ and succeeded in having $B$ adopt
it; or one of the honest parties (potentially $B$) was able to mine a block
which was subsequently adopted by $B$. In either case, the adversary continues
with the strategy by inspecting $\chain[-1]$ and acting accordingly.


\aknote{ Consider this alternative attack and analysis below: starting at the genesis block,
$\mathcal{A}$ mines continuously and
in private blocks of level strictly less than $\mu$ (i.e., it discards blocks of level $\mu$ or higher). Whenever honest parties broadcast a block $b$, $\mathcal{A}$ operates as follows:
(i) if there is a private block $b'$ of chain height equal to $b$,  release that block
in the network with rushing delivery (i.e.,
having $b'$ delivered ahead of $b$ for all honet parties) and continue mining
on top of the private block $b''$ that  in the private chain;
(ii) otherwise, adopt $b$ and continue mining on top of $b$.

Observe that the above attack matches the optimal chain quality
attacker of \cite{GKL}
with relative power $2^{-\mu}\cdot t/n$  (we assume without loss of generality
that $t$ is a multiple of $2^\mu$).
Using the chain quality theorem we have that honest parties will get at most
$(n- t 2^{-\mu+1}) / (n-t2^{-\mu}) + \delta' $ honest blocks in any of their chains
for some suitable constant $\delta'$ and all the remaining blocks are
adversarial and hence of a super level less than $\mu$.
Of the honest blocks in an honest party's chain,
the expected rate of $\mu$-level superblocks
will be  $2^{-\mu} ((n- t 2^{-\mu+1}) / (n-t2^{-\mu})  + \delta') $.
Now in order to break chain superquality it suffices to break
$\mu$-local-goodness of the chain $\chain$ of an honest party,
which is violated as long as
$  2^{-\mu} ( (n-  t2^{-\mu+1} ) / (n-t2^{-\mu}) + \delta') \leq (1-\delta) 2^{-\mu}$
XXX this needs some cleanup; but should work as long as adversary has suitably
high hashing power. The corresponding backbone paragraph is outdated (it was
left from the  pre-typical era) so i am looking into
 fixing it as well XXX.

 What is important though from this attack is the following:
 the expected superchain growth  of level $\mu$ of an honest party's  chain will drop from
 the expected rate of $2^{-\mu}$ in the all honest case,
 to $\frac{n- t2^{-\mu+1}}{n-t2^{-\mu}} 2^{-\mu}$.
 This could be close to $1/3$ for level $\mu=1$, and in general
 close to $\frac{2^{\mu} - 1}{2^{\mu+1} - 1}$.
}
Assume now that an honestly-generated $\mu$-superblock has been adopted by $B$
at position $\chain_B[i]$ at some round $r$. Let us now examine the probability
that $\chain_B[i]$ will remain a $\mu$-superblock in the long run. Suppose $r' >
r$ is the first round after $r$ during which any block is generated.
% Let
% $f = 1 - (1 - p)^{q(n-t)}$ and $f' = 1 - (1 - p)^{qt}$. Then the probability
% that only the adversary while the probability \cite{backbone}.
It is clear
that $\mathcal{A}$ will succeed in this attack with non-negligible probability
\dznote{Why? TODO: Calculate exact probability}
and cause $B$ to abandon the $\mu$-superblock from their adoped chain.
Therefore, there will be some $\delta$ such that the adversary will be able to
cause such variance with non-negligible probability in $m$.
\dznote{Why? TODO: Write out as theorem and prove}

\subsection{From chain superquality attacks to double-spending}

\aknote{We next show how the superquality attacker can be turned
to a double spending attacker against a verifier that accepts
proofs with the sole criterion that a superchain of a certain level
has length $m$.
The attacker divides its mining power between two parallel processes.
In the first process it runs the chain superquality attack of the previous
section for a specific superlevel $\mu$. In the second process,
it operates as a standard Nakamoto double-spending
attacker but its objective is to overcome the honest parties' chain
at superchain level $\mu$ (in contrast, say,  to the standard
Nakamoto double spending attacker that operates on level $\mu=0$).
The rationale behind the above strategy is that the superchain growth
of honest parties' has been slowed down due to the superchain quality
attack component, and thus the double spending attacker that targets
level $\mu$ has an unanticipated advantage.
}

Extending the above attack, we now modify the superquality attacker into an
attacker, $\mathcal{A}$ that can cause a double spending attack in the proof of
proof-of-work construction. As before, $\mathcal{A}$ targets the proofs
generated by the honest party $B$ by violating $\mu$-superquality in $B$'s
adopted chain. $\mathcal{A}$ begins by remaining idle until a certain chosen
block $b$. After block $b$ is produced, $\mathcal{A}$ starts mining a secret
chain which forks off from $b$ akin to a selfish mining attacker~\cite{selfish}.
The adversary performs a normal spending transaction on the honestly adopted
blockchain and has it confirmed in the block immediately following block $b$.
They also produce a double spending transaction which they secretly confirm in
their secret chain in the block immediately following block $b$.

$\mathcal{A}$ keeps extending their own secret chain as usual. However, whenever
a $\mu$-superblock is adopted by $B$, the adversary temporarily pauses mining in
their secret chain and uses the previous attack to harm $\mu$-superquality of
$B$'s adopted chain.

Observe now that $\mu$-superquality in $B$'s adopted chain is harmed the same as
before, because $\mathcal{A}$ only works on their secret chain whenever they
would remain idle in the previous attack. However, for sufficiently large $\mu$,
the cases needed to suspend working on the secret chain will be small, and hence
$\mathcal{A}$'s chain will grow with a sufficient rate \dznote{TODO: Make this
more precise}. $\mathcal{A}$'s chain $\mu$-superquality is going to be preserved
\dznote{TODO: Explain why?}. Therefore, with sufficient computing power
\dznote{TODO: Make this more precise}, $\mathcal{A}$ will be able to generate
$m$ $\mu$-superblocks in their secret chain before the honest parties are able
to produce $m$ $\mu$-superblocks. Using these in her PoPoW, $\mathcal{A}$ will
be able to win against the PoPoW produced by $B$.

\textbf{Remark.} It is worth isolating the mistake in the security proof from
the interactive construction paper \cite{KLS}. Suppose player $B$ is honest and
player $\mathcal{A}$ is adversarial and suppose $b$, the LCA block, was honestly
generated and suppose that the superchain comparison happens at level $\mu$.
Their proof then argues that there will have been more honestly- than
adversarially-generated $\mu$-superblocks after block $b$. Nevertheless, we
observe that the mere fact that there have been more honestly- than
adversarially-generated $\mu$-superblocks after $b$ does not imply that
$|\overline\pi_\mathcal{A}[\mu]\{b:\}| \leq |\overline\pi_B[\mu]\{b:\}|$. The
reason is that some of these superblocks could belong to blocktree forks that
have been abandoned by $B$. Thus, the security conclusion does not follow.

Regardless, their basic argument and construction is what will use as a basis
for constructing a system that is both provably secure and succinct under the
same assumptions, albeit requiring a more complicated proof structure to obtain
security.
