\section{An attack against proofs of proof-of-work}

In this section, we revisit the construction for interactive proofs of
proof-of-work from \cite{KLS} and its security.  Their
construction is the starting point and  basis for our
non-interactive proofs of proof-of-work. We show that the construction
is susceptible to  a double-spending attack even in the case the adversary
controls a minority of the parties (hashing power).  We recommend that the
interested reader skims through the KLS~\cite{KLS} work to fully understand the
scheme and the  attacks that follow.

We first show that a powerful attacker can break chain superquality with
non-negligible probability and we construct a concrete attack based on this
observation. Maintaining chain superquality is not in the original
security model. However, we show how the property affects the security  of the
underlying proofs. We do this by presenting a concrete attack against their
scheme that allows an adversary to double spend with non-negligible probability.
This attack works under the assumption that the adversary has sufficient hashing
power, which is still below $50\%$.

\subsection{The interactive proofs of proof-of-work protocol}

% \dznote{Include shortened pseudocode for KLS construction}

The interactive proof of
proof-of-work of \cite{KLS} operates as follows.
The core of the verification process is distinguishing between
two candidate proofs $(\pi_A, \chi_A)$  and $(\pi_B, \chi_B)$.
In case  $\pi_A  = \pi_B$, the decision can be drawn immediately (without interaction) based
on $\chi_A,\chi_B$.
Otherwise, the verifier queries the two provers for their claimed
anchored superchains $\pi_A\upchain^\mu$ and $\pi_B\upchain^\mu$ at a
certain level $\mu$. The verifier starts querying at the highest possible level
$\mu$ and successively descends until level $\mu$ is sufficiently low such that
the lowest common ancestor block $b = LCA(\pi_A\upchain^\mu,
\pi_B\upchain^\mu)$ of the two provers' $\mu$-superchains is $m$ blocks deep
for at least one of the provers. More specifically, the querying stops at such
$\mu$ when $max(|\pi_A\upchain^\mu\{b:\}|, |\pi_B\upchain^\mu\{b:\}|) \geq
m$ holds. Subsequently, the winner is designated as the prover with the most
blocks after $b$ at that level. More precisely, the winner is $A$ if
$|\pi_A\upchain^\mu\{b:\}| \geq |\pi_B\upchain^\mu\{b:\}|)$ and otherwise it
is $B$. The communication overhead is  reduced by only requesting
blocks after the purported LCA.

\subsection{Attacking chain superquality}
\label{subsec.superquality-attack}

We present next a general attack against chain superquality
that works against any blockchain protocol that incorporates
the interlink data structure, cf. Section~\ref{sec.interlink}.
Observe that in expectation, in a
completely honest setting with random network scheduling, the honestly adopted
chain will be $\mu$-locally-good for sufficiently long subchains everywhere, as
proven in Lemma~\ref{lem.localgood}.

Let us now construct an adversary to break this property at level $\mu$. Suppose
the adversary $\mathcal{A}$ controls a portion of the hashing power equal to
$t/n$. $\mathcal{A}$ then works as follows. Assume she wants to attack the
honest party $B$ in order to have him adopt a chain $\chain_B$ which has a bad
distribution of superblocks, i.e. such that local goodness is violated in some
sufficiently long subchain. She continuously determines the current chain
$\chain_B$ adopted by $B$ (as honest chain adoption is deterministic, this can
be easily done by inspecting the messages exchanged by the honest parties). The
adversary starts playing after $|\chain_B| \geq 2$. If
$\textit{level}(\chain_B[-1]) < \mu$, then $\mathcal{A}$ remains idle. However,
if $\textit{level}(\chain_B[-1]) \geq \mu$, then $\mathcal{A}$ attempts to mine
an adversarial block $b$ on top of $\chain_B[-2]$. If she is successful, then she
attempts to mine another block $b'$ on top of $b$. If she is successful, she
broadcasts both blocks $b, b'$. The adversarial mining continues until party $B$
adopts a new chain, which can be due to two reasons: Either the adversary
managed to successfully mine $b, b'$ on top of $\chain_B[-2]$ and succeeded in
having $B$ adopt it; or one of the honest parties (potentially $B$) was able to
mine a block which was subsequently adopted by $B$. In either case, the
adversary continues with the strategy by inspecting $\chain[-1]$ and acting
accordingly.


%  \aknote{ Consider this alternative attack and analysis below: starting at the genesis block,
%  $\mathcal{A}$ mines continuously and
%  in private blocks of level strictly less than $\mu$ (i.e., it discards blocks of level $\mu$ or higher). Whenever honest parties broadcast a block $b$, $\mathcal{A}$ operates as follows:
%  (i) if there is a private block $b'$ of chain height equal to $b$,  release that block
%  in the network with rushing delivery (i.e.,
%  having $b'$ delivered ahead of $b$ for all honet parties) and continue mining
%  on top of the private block $b''$ that  in the private chain;
%  (ii) otherwise, adopt $b$ and continue mining on top of $b$.
%
%  Observe that the above attack matches the optimal chain quality
%  attacker of \cite{GKL}
%  with relative power $2^{-\mu}\cdot t/n$  (we assume without loss of generality
%  that $t$ is a multiple of $2^\mu$).
%  Using the chain quality theorem we have that honest parties will get at most
%  $(n- t 2^{-\mu+1}) / (n-t2^{-\mu}) + \delta' $ honest blocks in any of their chains
%  for some suitable constant $\delta'$ and all the remaining blocks are
%  adversarial and hence of a super level less than $\mu$.
%  Of the honest blocks in an honest party's chain,
%  the expected rate of $\mu$-level superblocks
%  will be  $2^{-\mu} ((n- t 2^{-\mu+1}) / (n-t2^{-\mu})  + \delta') $.
%  Now in order to break chain superquality it suffices to break
%  $\mu$-local-goodness of the chain $\chain$ of an honest party,
%  which is violated as long as
%  $  2^{-\mu} ( (n-  t2^{-\mu+1} ) / (n-t2^{-\mu}) + \delta') \leq (1-\delta) 2^{-\mu}$
%  XXX this needs some cleanup; but should work as long as adversary has suitably
%  high hashing power. The corresponding backbone paragraph is outdated (it was
%  left from the  pre-typical era) so i am looking into
%   fixing it as well XXX.
%
%   What is important though from this attack is the following:
%   the expected superchain growth  of level $\mu$ of an honest party's  chain will drop from
%   the expected rate of $2^{-\mu}$ in the all honest case,
%   to $\frac{n- t2^{-\mu+1}}{n-t2^{-\mu}} 2^{-\mu}$.
%   This could be close to $1/3$ for level $\mu=1$, and in general
%   close to $\frac{2^{\mu} - 1}{2^{\mu+1} - 1}$.
%  }
Assume now that an honestly-generated $\mu$-superblock has been adopted by $B$
at position $\chain_B[i]$ at some round $r$. Let us now examine the probability
that $\chain_B[i]$ will remain a $\mu$-superblock in the long run. Suppose $r' >
r$ is the first round after $r$ during which any block is generated.
% Let
% $f = 1 - (1 - p)^{q(n-t)}$ and $f' = 1 - (1 - p)^{qt}$. Then the probability
% that only the adversary while the probability \cite{backbone}.
It is clear
that $\mathcal{A}$ will succeed in this attack with non-negligible probability
% \dznote{Why? TODO: Calculate exact probability}
and cause $B$ to abandon the $\mu$-superblock from their adopted chain.
Therefore, there will be some $\delta$ such that the adversary will be able to
cause such variance with non-negligible probability in $m$. This suffices to
show that superquality is harmed by this attack.
% \dznote{Why? TODO: Write out as theorem and prove}
To avoid repeating ourselves, the full understanding of the probabilities for
this attack will become clear in the next section where the double-spending
attack is explored in detail.

\subsection{From chain superquality attacks to double-spending}

%  \aknote{We next show how the superquality attacker can be turned
%  to a double spending attacker against a verifier that accepts
%  proofs with the sole criterion that a superchain of a certain level
%  has length $m$.
%  The attacker divides its mining power between two parallel processes.
%  In the first process it runs the chain superquality attack of the previous
%  section for a specific superlevel $\mu$. In the second process,
%  it operates as a standard Nakamoto double-spending
%  attacker but its objective is to overcome the honest parties' chain
%  at superchain level $\mu$ (in contrast, say,  to the standard
%  Nakamoto double spending attacker that operates on level $\mu=0$).
%  The rationale behind the above strategy is that the superchain growth
%  of honest parties' has been slowed down due to the superchain quality
%  attack component, and thus the double spending attacker that targets
%  level $\mu$ has an unanticipated advantage.
%
%  In a simplified model, consider an attacker that produces
%  blocks with probability $p<1/2$ while honest parties following the protocol
%  produce blocks with probability $1-p$.
%  Whenever the honest parties produce a block that belongs to level $\mu$
%  (an event that happens with probability $(1-p)2^{-\mu}$, the
%  attacker attempts to mine two blocks of level less than $\mu$ and
%  use them to remove the $\mu$-level superblock in the chain of the
%  honest parties; this happens with probability $(1-2^{-\mu})^2p^2$.
%
%  From this we can observe that the $\mu$-level superchain
%   of the honest parties will grow with a slightly smaller rate:
%  instead of $2^{-\mu} (1-p)$ that is the case when the adversary
%  simply mines a private blockchain, it is now $2^{-\mu} (1-p) (1 - (1-2^{-\mu})^2p^2)$.
%  At the same time  the $\mu$-level superchain of the adversary will grow with rate
%  $2p^2 2^{-\mu} + 2 p(1-p)2^{-\mu}(1-2^{-\mu}) $.
%  }

Extending the above attack, we now modify the superquality attacker into an
attacker, $\mathcal{A}$, that can cause a double spending attack in the proof of
proof-of-work construction. As before, $\mathcal{A}$ targets the proofs
generated by the honest party $B$ by violating $\mu$-superquality in $B$'s
adopted chain. $\mathcal{A}$ begins by remaining idle until a certain chosen
block $b$. After block $b$ is produced, $\mathcal{A}$ starts mining a secret
chain which forks off from $b$ akin to a selfish mining attacker~\cite{selfish}.
The adversary performs a normal spending transaction on the honestly adopted
blockchain and has it confirmed in the block immediately following block $b$.
They also produce a double spending transaction which they secretly confirm in
their secret chain in the block immediately following block $b$.

$\mathcal{A}$ keeps extending their own secret chain as usual. However, whenever
a $\mu$-superblock is adopted by $B$, the adversary temporarily pauses mining in
their secret chain and devotes her mining power to harm the $\mu$-superquality
of $B$'s adopted chain. Intuitively, for large enough $\mu$, the time spent
trying to harm superquality will be limited, because the probability of a
$\mu$-superblock occurring will be small. Therefore, the adversary's superchain
quality will be larger than the honest parties' superchain quality (which will
be harmed by the adversary) and therefore, even though the adversary's
0-level blockchain will be shorter than the honest parties' 0-level blockchain,
the adversary's $\mu$-superchain will be longer than the honest parties'
$\mu$-superchain.

We now proceed to calculate the attack probabilities precisely and in formal
detail. We simplify the above attack to ease the formal probabilistic analysis.
The attack is parameterized by two parameters $r$ and $\mu$ which are picked by
the adversary. $\mu$ will be the superblock level at which the adversary will
attempt to produce a proof longer than the honest proof. The modified attack
works precisely as follows: Without loss of generality, we fix block $b$ to be
the Genesis block. The adversary always mines on the secret chain which forks
off from genesis, unless a \textit{superblock generation event} occurs. If a
superblock generation event occurs, then the adversary pauses mining on the
secret chain and attempts a \textit{block suppression attack} on the honest
chain. The adversary devotes exactly $r$ rounds to this suppression attack and
then resumes mining on their secret chain. Our goal is to show that, despite
this simplification (of fixing $r$) which is harmful to the adversary, the
probability of a successful attack is non-negligible for certain (reasonable)
values of the protocol parameters.

A superblock generation event is detected by the adversary by monitoring the
network. Whenever an honest party diffuses an honestly-generated
$\mu$-superblock at the end of a given round $r_1$, then the superblock
generation event will have occurred and the adversary will starting devoting
their mining power to block suppression starting from the next round.

A block suppression attack works as follows. Let $B$ be the honestly generated
$\mu$-superblock which was diffused at the end of the previous round. If the
round was not uniquely successful, let $B$ be any of the diffused
honestly-generated $\mu$-superblocks. Let $B$ be the tip of an honest chain
$\chain_B$. The adversary first mines on top of $\chain_B[-2]$. If she is
successful in mining a block $B'$, then she continues extending the chain ending
at $B'$ (to mine $B''$ and so on). The value $r$ is fixed, and so the adversary
devotes exactly $r$ rounds to this whole process; the adversary will keep mining
on top of $\chain_B[-2]$ (or one of the adversarially-generated extensions of
it) for exactly $r$ rounds, regardless of whether $B'$ or $B''$ have been found.
At the same time, the honest parties will be mining on top of $B$ (or a
competing block in the case of a non-uniquely successful round). Again, further
successful block diffusion by the honest parties shall not affect that the
adversary is going to spend exactly $r$ rounds for suppression.

Having laid out the attack scenario precisely, we are ready to prove that it
will succeed with non-negligible probability. In fact, as we will see, perhaps
surprisingly, it will succeed with overwhelming probability for the right choice
of protocol values.

\begin{restatable}[Double-spending attack]{theorem}{restateThmAttack}
There exist parameters $p, n, t, q,  \mu, \delta$, with $t\leq (1-\delta)(n-t)$,
and a double spending attack against KLS PoPoW that succeeds with overwhelming
probability.
\end{restatable}
\ifonecolumn
\import{./}{proofs/attack.tex}
\fi

\textbf{Remark.} It is worth isolating the mistake in the security proof from
the interactive construction paper \cite{KLS}. Suppose player $B$ is honest and
player $\mathcal{A}$ is adversarial and suppose $b$, the LCA block, was honestly
generated and suppose that the superchain comparison happens at level $\mu$.
Their proof then argues that there will have been more honestly- than
adversarially-generated $\mu$-superblocks after block $b$. Nevertheless, we
observe that the mere fact that there have been more honestly- than
adversarially-generated $\mu$-superblocks after $b$ does not imply that
$|\overline\pi_\mathcal{A}[\mu]\{b:\}| \leq |\overline\pi_B[\mu]\{b:\}|$. The
reason is that some of these superblocks could belong to blocktree forks that
have been abandoned by $B$. Thus, the security conclusion does not follow.

Regardless, their basic argument and construction is what we will use as a basis
for constructing a system that is both provably secure and succinct under the
same assumptions, albeit requiring a more complicated proof structure to obtain
security.
