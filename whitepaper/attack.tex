\section{An attack against proofs of proof-of-work}

In this section, we revisit the construction for interactive proofs of
proof-of-work from \cite{KLS} and its security.  Their 
construction is the starting point and  basis for our
non-interactive proofs of proof-of-work. We show that the construction 
is susceptible to  a double-spending attack even in the case it controls
 a minority of the parties (hashing power). 
We recommend that the interested reader skims
through the KLS~\cite{KLS} work to fully understand the scheme and the 
attacks that follow.

We first show that a powerful attacker can break chain superquality with
non-negligible probability and we construct a concrete attack based on it.
Even though maintaining chain superquality is not in the original security model, 
we show how the property affects the security  of the underlying proofs
by presenting a
concrete attack against their scheme that allows an adversary to double spend
with non-negligible probability assuming sufficient hashing power (that is
still below $50\%$).

\subsection{The interactive proofs of proof-of-work protocol}

% \dznote{Include shortened pseudocode for KLS construction}

The interactive proof of
proof-of-work of \cite{KLS} operates as follows.
The core of the verification process is distinguishing between
two candidate proofs $(\pi_A, \chi_A)$  and $(\pi_B, \chi_B)$.
In case  $\pi_A  = \pi_B$, the decision can be drawn immediately (without interaction) based
on $\chi_A,\chi_B$.
Otherwise, the verifier queries the two provers for their claimed
anchored superchains $\pi_A\upchain^\mu$ and $\pi_B\upchain^\mu$ at a
certain level $\mu$. The verifier starts querying at the highest possible level
$\mu$ and successively descends until level $\mu$ is sufficiently low such that
the lowest common ancestor block $b = LCA(\pi_A\upchain^\mu,
\pi_B\upchain^\mu)$ of the two provers' $\mu$-superchains is $m$ blocks deep
for at least one of the provers. More specifically, the querying stops at such
$\mu$ when $max(|\pi_A\upchain^\mu\{b:\}|, |\pi_B\upchain^\mu\{b:\}|) \geq
m$ holds. Subsequently, the winner is designated as the prover with the most
blocks after $b$ at that level. More precisely, the winner is $A$ if
$|\pi_A\upchain^\mu\{b:\}| \geq |\pi_B\upchain^\mu\{b:\}|)$ and otherwise it
is $B$. The communication overhead is  reduced by only requesting
blocks after the purported LCA.

\subsection{Attacking chain superquality}
\label{subsec.superquality-attack}

We present next a general attack against chain superquality
that works against any blockchain protocol that incorporates
the interlink data structure, cf. Section~\ref{sec.interlink}.
Observe that in expectation, in a
completely honest setting with random network scheduling, the honestly adopted
chain will be $\mu$-locally-good for sufficiently long subchains everywhere as
proven in Lemma~\ref{lem.localgood}.

Let us now construct an adversary to break this property at level $\mu$. Suppose
the adversary $\mathcal{A}$ controls a portion of the hashing power equal to
$t/n$. $\mathcal{A}$ then works as follows. Assume she wants to attack the
honest party $B$ in order to have him adopt a chain $\chain_B$ which has a bad
distribution of superblocks, i.e. such that local goodness is violated in some
sufficiently long subchain. She continuously determines the current chain
$\chain_B$ adopted by $B$ (as honest chain adoption is deterministic, this can
be easily done by inspecting the messages exchanged by the honest parties). The
adversary starts playing after $|\chain_B| \geq 2$. If
$\textit{level}(\chain_B[-1]) < \mu$, then $\mathcal{A}$ remains idle. However,
if $\textit{level}(\chain_B[-1]) \geq \mu$, then $\mathcal{A}$ attempts to mine
an adversarial block $b$ on top of $\chain_B[-2]$. If she is successful, then she
attempts to mine another block $b'$ on top of $b$. If she is successful, she
broadcasts both blocks $b, b'$. The adversarial mining continues until party $B$
adopts a new chain, which can be due to two reasons: Either the adversary
managed to successfully mine $b, b'$ on top of $\chain_B[-2]$ and succeeded in
having $B$ adopt it; or one of the honest parties (potentially $B$) was able to
mine a block which was subsequently adopted by $B$. In either case, the
adversary continues with the strategy by inspecting $\chain[-1]$ and acting
accordingly.


%  \aknote{ Consider this alternative attack and analysis below: starting at the genesis block,
%  $\mathcal{A}$ mines continuously and
%  in private blocks of level strictly less than $\mu$ (i.e., it discards blocks of level $\mu$ or higher). Whenever honest parties broadcast a block $b$, $\mathcal{A}$ operates as follows:
%  (i) if there is a private block $b'$ of chain height equal to $b$,  release that block
%  in the network with rushing delivery (i.e.,
%  having $b'$ delivered ahead of $b$ for all honet parties) and continue mining
%  on top of the private block $b''$ that  in the private chain;
%  (ii) otherwise, adopt $b$ and continue mining on top of $b$.
%
%  Observe that the above attack matches the optimal chain quality
%  attacker of \cite{GKL}
%  with relative power $2^{-\mu}\cdot t/n$  (we assume without loss of generality
%  that $t$ is a multiple of $2^\mu$).
%  Using the chain quality theorem we have that honest parties will get at most
%  $(n- t 2^{-\mu+1}) / (n-t2^{-\mu}) + \delta' $ honest blocks in any of their chains
%  for some suitable constant $\delta'$ and all the remaining blocks are
%  adversarial and hence of a super level less than $\mu$.
%  Of the honest blocks in an honest party's chain,
%  the expected rate of $\mu$-level superblocks
%  will be  $2^{-\mu} ((n- t 2^{-\mu+1}) / (n-t2^{-\mu})  + \delta') $.
%  Now in order to break chain superquality it suffices to break
%  $\mu$-local-goodness of the chain $\chain$ of an honest party,
%  which is violated as long as
%  $  2^{-\mu} ( (n-  t2^{-\mu+1} ) / (n-t2^{-\mu}) + \delta') \leq (1-\delta) 2^{-\mu}$
%  XXX this needs some cleanup; but should work as long as adversary has suitably
%  high hashing power. The corresponding backbone paragraph is outdated (it was
%  left from the  pre-typical era) so i am looking into
%   fixing it as well XXX.
%
%   What is important though from this attack is the following:
%   the expected superchain growth  of level $\mu$ of an honest party's  chain will drop from
%   the expected rate of $2^{-\mu}$ in the all honest case,
%   to $\frac{n- t2^{-\mu+1}}{n-t2^{-\mu}} 2^{-\mu}$.
%   This could be close to $1/3$ for level $\mu=1$, and in general
%   close to $\frac{2^{\mu} - 1}{2^{\mu+1} - 1}$.
%  }
Assume now that an honestly-generated $\mu$-superblock has been adopted by $B$
at position $\chain_B[i]$ at some round $r$. Let us now examine the probability
that $\chain_B[i]$ will remain a $\mu$-superblock in the long run. Suppose $r' >
r$ is the first round after $r$ during which any block is generated.
% Let
% $f = 1 - (1 - p)^{q(n-t)}$ and $f' = 1 - (1 - p)^{qt}$. Then the probability
% that only the adversary while the probability \cite{backbone}.
It is clear
that $\mathcal{A}$ will succeed in this attack with non-negligible probability
% \dznote{Why? TODO: Calculate exact probability}
and cause $B$ to abandon the $\mu$-superblock from their adoped chain.
Therefore, there will be some $\delta$ such that the adversary will be able to
cause such variance with non-negligible probability in $m$. This suffices to
show that superquality is harmed by this attack.
% \dznote{Why? TODO: Write out as theorem and prove}
To avoid repeating ourselves, the full understanding of the probabilities for
this attack will become clear in the next section where the double-spending
attack is explored in detail.

\subsection{From chain superquality attacks to double-spending}

%  \aknote{We next show how the superquality attacker can be turned
%  to a double spending attacker against a verifier that accepts
%  proofs with the sole criterion that a superchain of a certain level
%  has length $m$.
%  The attacker divides its mining power between two parallel processes.
%  In the first process it runs the chain superquality attack of the previous
%  section for a specific superlevel $\mu$. In the second process,
%  it operates as a standard Nakamoto double-spending
%  attacker but its objective is to overcome the honest parties' chain
%  at superchain level $\mu$ (in contrast, say,  to the standard
%  Nakamoto double spending attacker that operates on level $\mu=0$).
%  The rationale behind the above strategy is that the superchain growth
%  of honest parties' has been slowed down due to the superchain quality
%  attack component, and thus the double spending attacker that targets
%  level $\mu$ has an unanticipated advantage.
%
%  In a simplified model, consider an attacker that produces
%  blocks with probability $p<1/2$ while honest parties following the protocol
%  produce blocks with probability $1-p$.
%  Whenever the honest parties produce a block that belongs to level $\mu$
%  (an event that happens with probability $(1-p)2^{-\mu}$, the
%  attacker attempts to mine two blocks of level less than $\mu$ and
%  use them to remove the $\mu$-level superblock in the chain of the
%  honest parties; this happens with probability $(1-2^{-\mu})^2p^2$.
%
%  From this we can observe that the $\mu$-level superchain
%   of the honest parties will grow with a slightly smaller rate:
%  instead of $2^{-\mu} (1-p)$ that is the case when the adversary
%  simply mines a private blockchain, it is now $2^{-\mu} (1-p) (1 - (1-2^{-\mu})^2p^2)$.
%  At the same time  the $\mu$-level superchain of the adversary will grow with rate
%  $2p^2 2^{-\mu} + 2 p(1-p)2^{-\mu}(1-2^{-\mu}) $.
%  }

Extending the above attack, we now modify the superquality attacker into an
attacker, $\mathcal{A}$, that can cause a double spending attack in the proof of
proof-of-work construction. As before, $\mathcal{A}$ targets the proofs
generated by the honest party $B$ by violating $\mu$-superquality in $B$'s
adopted chain. $\mathcal{A}$ begins by remaining idle until a certain chosen
block $b$. After block $b$ is produced, $\mathcal{A}$ starts mining a secret
chain which forks off from $b$ akin to a selfish mining attacker~\cite{selfish}.
The adversary performs a normal spending transaction on the honestly adopted
blockchain and has it confirmed in the block immediately following block $b$.
They also produce a double spending transaction which they secretly confirm in
their secret chain in the block immediately following block $b$.

$\mathcal{A}$ keeps extending their own secret chain as usual. However, whenever
a $\mu$-superblock is adopted by $B$, the adversary temporarily pauses mining in
their secret chain and devotes her mining power to harm the $\mu$-superquality
of $B$'s adopted chain. Intuitively, for large enough $\mu$, the time spent
trying to harm superquality will be limited, because the probability of a
$\mu$-superblock occurring will be small. Therefore, the adversary's superchain
quality will be larger than the honest parties' superchain quality (which will
be harmed by the adversary) and therefore, even though the adversary's
0-level blockchain will be shorter than the honest parties' 0-level blockchain,
the adversary's $\mu$-superchain will be longer than the honest parties'
$\mu$-superchain.

We now proceed to calculate the attack probabilities precisely and in formal
detail. We simplify the above attack to ease the formal probabilistic analysis.
The attack is parameterized by two parameters $r$ and $\mu$ which are picked by
the adversary. $\mu$ will be the superblock level at which the adversary will
attempt to produce a proof longer than the honest proof. The modified attack
works precisely as follows: Without loss of generality, we fix block $b$ to be
the Genesis block. The adversary always mines on the secret chain which forks
off from genesis, unless a \textit{superblock generation event} occurs. If a
superblock generation event occurs, then the adversary pauses mining on the
secret chain and attempts a \textit{block suppression attack} on the honest
chain. The adversary devotes exactly $r$ rounds to this suppression attack and
then resumes mining on their secret chain. Our goal is to show that, despite
this simplification (of fixing $r$) which is harmful to the adversary, the
probability of a successful attack is non-negligible for certain (reasonable)
values of the protocol parameters.

A superblock generation event is detected by the adversary by monitoring the
network. Whenever an honest party diffuses an honestly-generated
$\mu$-superblock at the end of a given round $r_1$, then the superblock
generation event will have occurred and the adversary will starting devoting
their mining power to block suppression starting from the next round.

A block suppression attack works as follows. Let $B$ be the honestly generated
$\mu$-superblock which was diffused at the end of the previous round. If the
round was not uniquely successful, let $B$ be any of the diffused
honestly-generated $\mu$-superblocks. Let $B$ be the tip of an honest chain
$\chain_B$. The adversary first mines on top of $\chain_B[-2]$. If she is
successful in mining a block $B'$, then she continues extending the chain ending
at $B'$ (to mine $B''$ and so on). The value $r$ is fixed, and so the adversary
devotes exactly $r$ rounds to this whole process; the adversary will keep mining
on top of $\chain_B[-2]$ (or one of the adversarially-generated extensions of
it) for exactly $r$ rounds, regardless of whether $B'$ or $B''$ have been found.
At the same time, the honest parties will be mining on top of $B$ (or a
competing block in the case of a non-uniquely successful round). Again, further
successful block diffusion by the honest parties shall not affect that the
adversary is going to spend exactly $r$ rounds for suppression.

Having laid out the attack scenario precisely, we are ready to prove that it
will succeed with non-negligible probability. In fact, as we will see, perhaps
surprisingly, it will succeed with overwhelming probability for the right choice
of protocol values.

\begin{theorem}
There exist parameters $p, n, t, q, r, \mu, \delta$, with $(n-t) \geq t(1+\delta)$,
and a double spending attack against KLS PoPoW that succeeds with overwhelming
probability.
\end{theorem}
\begin{proof}
Recall that in the backbone notation $n$ denotes the total number of parties,
$t$ denotes the number of adversarial parties, $q$ denotes the number of the
random oracle queries allowed per party per round and $p$ is the probability that
one random oracle query will be successful and remember that $p = T / 2^\kappa$
where $T$ is the mining target and $\kappa$ is the security parameter (or hash
function bit count). Then $f$ denotes the probability that a given round is
successful and we have that $f = 1 - (1 - p)^{q(n-t)}$. Recall further that a
requirement of the backbone protocol is that the honest majority assumption is
satisfied, that is that $t \leq (1 - \delta)(n - t)$ were $\delta \geq 2f +
3\epsilon$, where $\epsilon \in (0, 1)$ is an arbitrary small constant describing
the quality of the concentration of the random variables.

Denote $\alpha_\mathcal{A}$ the secret chain generated by the adversary and
$\alpha_B$ the honest chain belonging to any honest party. We will show that
for certain protocol values we have that
$\Pr[|\alpha_\mathcal{A}\upchain^\mu| \geq |\alpha_B\upchain^\mu|]$ is
overwhelming.

Assume that, to the adversary's harm and to simplify the analysis, the adversary
plays at beginning of every round and does not perform adversarial scheduling.
At the beginning of the round when it is the adversary's turn to play, she has
access to the blocks diffused during the previous round by the honest parties.

First, observe that at the beginning of each round, the adversary finds herself
in one of two different situations: Either she has been forced into an
$r$-round-long period of suppression, or she is not in that period. If she is
within that period, she blindly performs the suppression attack without regard
for the state of the world. If she is not within that period, then she must
initially observe the blocks diffused at the end of the previous round by the
honest parties. Call these rounds during which the diffused data must be
examined by the adversary \textit{decision rounds}. Let there be $\omega$
decision rounds in total. In each such decision round, it is possible that the
adversary discovers a diffused $\mu$-superblock and therefore decides that a
suppression attack must be performed starting with the current round. Call these
rounds during which this discovery is made by the adversary \textit{migration
rounds}. Let there be $y$ migration rounds in total. The adversary devotes the
migration round to performing the suppression attack as well as $r - 1$
non-migration rounds after the migration round. Call these rounds, including the
migration round, \textit{suppression rounds}. In the rest of the decision
rounds, the adversary will not find any $\mu$-superblocks diffused. Call these
\textit{secret chain rounds}; these are rounds where the adversary devotes her
queries to mining on the secret chain. Let there be $x$ secret chain rounds. If
the adversary devotes $\omega$ decision rounds to the attack in total, then
clearly we have that $\omega = x + y$. If the total number of rounds during
which the attack is running is $s$ then we also have that $s = x + ry$, because
for each migration round there are $r - 1$ non-decision rounds that follow.

We will analyze the honest and adversarial superchain lengths with respect to
$\omega$, which roughly corresponds to time (because note that $\omega \geq
s/r$, and so $\omega$ is proportional to the number of rounds).

Let us calculate the probability $p_{SB}$ (``superblock probability'') that a
decision round ends up being a migration round. Ignoring the negligible event
that there will be random oracle collisions, we have that $p_{SB} = (n -
t)qp2^{-\mu}$.

Given this, note that the decision taken at the beginning of each decision round
follows independent Bernoulli distributions with probability $p_{SB}$. Denote
$z_i$ the indicator random variable indicating whether the decision round was
a migration round. Therefore we can readily calculate the expectations for the
random variables $x$ and $y$, as $x = \omega - y$, $y = \sum_{i=1}^\omega z_i$.
We have $E[x] = (1 - p_{SB})\omega$ and $E[y] = p_{SB}\omega$. Applying a
Chernoff bound to the random variables $x$ and $y$, we observe that they will
attain values close to their mean for large $\omega$ and in particular
$\Pr[y \geq (1 + \delta)E[y]] \leq \exp(-\frac{\delta^2}{3} E[y])$ and similarly
$\Pr[x \leq (1 - \delta)E[x]] \leq \exp(-\frac{\delta^2}{2} E[x])$, which are
negligible in $\omega$.

Given that there will be $x$ secret chain rounds, we observe that the random
variable indicating the length of the secret adversarial superchain follows the
binomial distribution with $xtq$ repetitions and probability $p2^{-\mu}$. We
can now calculate the expected secret chain length as
$E[|\alpha_\mathcal{A}\upchain^\mu|] = xtqp2^{-\mu}$. Observe that in this
probability we have given the adversary the intelligence to continue using her
random oracle queries during a round even after a block has been found during a
round and not to wait for the next round. Applying a Chernoff bound, we obtain
that $\Pr[|\alpha_\mathcal{A}\upchain^\mu| \leq (1 -
\delta)E[|\alpha_\mathcal{A}\upchain^\mu|]] \leq
exp(-\frac{\delta^2}{2}E[|\alpha_\mathcal{A}\upchain^\mu|])$, which is
negligible in $\omega$ (because we know that with overwhelming probability $x >
(1 - \delta)(1 - p_{SB})\omega$).

It remains to calculate the behavior of the honest superchain.

Suppose that a migration round occurs during which at least one superblock $B$
is diffused. We will now calculate the probability $p_{sup}$ that the adversary
is able to suppress that block after $r$ rounds by performing the suppression
attack and cause all honest parties to adopt a chain not containing $B$.

One way for this to occur is if the adversary has generated exactly $2$ shallow
blocks (blocks which are not $\mu$-superblocks) after exactly $r$ rounds and the
honest parties having generated exactly $0$ blocks after exactly $r$ rounds.
This provides a lower bound for the probability, which is sufficient for our
purposes. Call ADV-WIN the event where the adversary has generated exactly $2$
shallow blocks after exactly $r$ rounds since the diffusion of $B$ and call
HON-LOSE the event where the honest parties have generated exactly $0$ blocks
after exactly $r$ rounds since the diffusion of $B$.

The number of blocks generated by the adversary after the diffusion of $B$
follows the binomial distribution with $r$ repetitions and probability $p_{LB}$,
where $p_{LB}$ denotes the probability that the adversary is able to produce a
shallow block (``low block probability'') during a single round. We have that
$p_{LB} = tqp(1 - 2^{-\mu})$. To evaluate $\Pr[\text{ADV-WIN}]$, we evaluate the
binomial distribution for $2$ successes to obtain $\Pr[\text{ADV-WIN}] =
\frac{r(r - 1)}{2} p_{LB}^2 (1 - p_{LB})^{r - 2}$. The number of blocks
generated by the honest parties after the diffusion of $B$ follows the binomial
distribution with $r$ repetitions and probability $f$. To evaluate
$\Pr[\text{HON-LOSE}]$, we evaluate the binomial distribution for $0$ successes
to obtain $\Pr[\text{HON-LOSE}] = (1 - f)^r$. Note that this is an upper bound
in the probability, in particular because there can be multiple
blocks during a non-uniquely successful round during which a $\mu$-superblock
was generated.

Then observe that the two events ADV-WIN and HON-LOSE are independent and
therefore
$p_{sup} =
 \Pr[\text{ADV-WIN}]\Pr[\text{HON-LOSE}] =
 \frac{r(r - 1)}{2} p_{LB}^2 (1 - p_{LB})^{r - 2}(1 - f)^r$.

Now that we have evaluated $p_{sup}$,  we will calculate the honest chain length
in two chunks: The superblocks generated and adopted by the honest parties
during secret chain rounds, $\chain_1$, and the superblocks generated and
adopted by the honest parties during suppression rounds, $\chain_2$ (and note
that these sets of blocks are not blockchains on their own).

$|\chain_1|$ is a random variable following the binomial distribution with
$s(n - t)q$ repetitions and probability $p2^{-\mu}(1 - p_{sup})$. In the
evaluation of this distribution, we give the honest parties the liberty to
belong to a mining pool and share mining information within a round, an
assumption which only makes matters for the adversary worse. We can now
calculate the expected length of $\chain_1$ to find
$E[|\chain_1|] = s(n - t)qp2^{-\mu}(1 - p_{sup})$. Applying a Chernoff bound, we
find that
$\Pr[|\chain_1| \geq (1 + \delta)E[|\chain_1|]]
\leq exp(-\frac{\delta^2}{3}E[|\chain_1|])$, which is negligible in $s$.

Finally, some additional $\mu$-superblocks could have been generated by the
honest parties while the adversary is spending $r$ rounds attempting to suppress
a previous $\mu$-superblock. These $\mu$-superblocks will be adopted in the case
the adversary fails to suppress the previous $\mu$-superblock. As the adversary
does not devote any decision rounds to these new $\mu$-superblocks, they will
never be suppressed if the previous $\mu$-superblock is not suppressed. We
collect these in the set $\chain_2$. To calculate $|\chain_2|$, observe that the
number of unsuppressed $\mu$-superblocks which caused an adversarial suppression
period is $|\chain_1|$. For each of these blocks, the honest parties spend $r$
rounds attempting to form further $\mu$-superblocks on top. The total number of
such attemps is $r|\chain_1|$. Therefore, the number of further honestly
generated $\mu$-superblocks attained during the $|\chain_1|$ different $r$-round
periods follows a binomial distribution with $|\chain_1|rq(n - t)$ repetitions
and probability $p2^{-\mu}$. Here we allow the honest parties to use repeated
queries within a round even after a shallow success and to work in a pool
to obtain an upper bound for the expectation. Therefore $E[|\chain_2|] =
|\chain_1|rq(n - t)p2^{-\mu}$ and applying a Chernoff bound we obtain that
$\Pr[|\chain_2| \geq (1 + \delta)E[|\chain_2|]] \leq
exp(-\frac{\delta}{3}E[|\chain_2|])$, which is negligible in $s$ and has a
quadratic error term. We deduce that $|\chain_2|$ will have a very small length
compared to the rest of the honest chain, as it is a vanishing term in $\mu$.

Concluding the calculation of the adversarial superchain, we get
$E[|\alpha_B\upchain^\mu|] = E[|\chain_1|] + E[|\chain_2|]$.

Finally, it remains to show that there exist values $p, n, t, q, r, \mu, \delta$
such that a $E[|\alpha_\mathcal{A}\upchain^\mu|] \geq (1 +
\delta)E[|\alpha_B\upchain^\mu|]$.

Using the values $p = 10^{-5}, q = 1, n =
1000, t = 489, \mu = 25, r = 200$, we observe that the honest majority
assumption is preserved. Replacing these values into the expectations formulae
above, we obtain $E[|\alpha_\mathcal{A}\upchain^\mu] \approx 1.457 * 10^{-10} *
\omega$ and $E[|\alpha_B\upchain^\mu] \approx 1.424 * 10^{-10} * \omega$, which
result to a constant gap $\delta$. Because the adversarial chain grows linearly
in $\omega$, the adversary only has to wait sufficient rounds for obtaining $m$
blocks to create a valid proof. Therefore, for these values, the adversary
will be able to generate a convincing PoPoW at some level $\mu$ which is longer
than the honest parties' proof, even when the adversary does not have a longer
underlying blockchain.
\qed
\end{proof}

\textbf{Remark.} It is worth isolating the mistake in the security proof from
the interactive construction paper \cite{KLS}. Suppose player $B$ is honest and
player $\mathcal{A}$ is adversarial and suppose $b$, the LCA block, was honestly
generated and suppose that the superchain comparison happens at level $\mu$.
Their proof then argues that there will have been more honestly- than
adversarially-generated $\mu$-superblocks after block $b$. Nevertheless, we
observe that the mere fact that there have been more honestly- than
adversarially-generated $\mu$-superblocks after $b$ does not imply that
$|\overline\pi_\mathcal{A}[\mu]\{b:\}| \leq |\overline\pi_B[\mu]\{b:\}|$. The
reason is that some of these superblocks could belong to blocktree forks that
have been abandoned by $B$. Thus, the security conclusion does not follow.

Regardless, their basic argument and construction is what we will use as a basis
for constructing a system that is both provably secure and succinct under the
same assumptions, albeit requiring a more complicated proof structure to obtain
security.
