\subsection{Succinctness}

We now prove that our construction produces succinct proofs.

\begin{restatable}{theorem}{restateThmFewLevels}
    \label{thm.few-levels}
    The number of superblock levels which have at least $m$ blocks are at most
    $\log(|S|)$, where $S$ is the set of all blocks produced, with overwhelming
    probability in $m$.
\end{restatable}

\begin{restatable}{theorem}{restateThmLargeExpansion}
    \label{thm.large-expansion}
    Given an honest $(\mu - 1)$-level superchain of at least $4m$ blocks,
    the respective $\mu$-level
    superchain which covers the same span of blocks has more than $m$ blocks,
    with overwhelming probability in $m$.
\end{restatable}

\begin{restatable}{corollary}{restateThmSmallSupport}
    \label{crly.small-support}
    Given an honest $\mu$-level superchain with $m$ blocks, the respective
    $(\mu - 1)$-level superchain spanning the same blocks has fewer than $4m$
    blocks, with overwhelming probability in $m$.
\end{restatable}

\begin{restatable}{theorem}{restateThmSuccinctness}
    \label{thm.succinctness}
    Non-interactive proofs-of-proof-of-work produced by honest provers on
    honest chains are succinct with the number of blocks bounded by $4m
    \log(|\chain|)$, with overwhelming probability in $m$.
\end{restatable}

% \subsection{Slow-down attacks}
%
% We have proven that honest provers working on honest chains always produce succinct
% proofs. We observe that this is an improvement over \cite{KLS}. We notice that
% their construction is vulnerable to an adversary which can cause honest provers
% working on honest chains to produce non-succinct proofs, and their succinctness
% argument works only in the optimistic case where all provers are honest. We
% term this a ``slow-down'' attack.
%
% The attack occurs when an adversarial prover produces a proof which includes
% some adversarially mined blocks. This adversarial proof is succinct, yet it
% causes honest provers working on completely honest chains to produce long
% proofs. Specifically, the adversary works as follows: After the honest chain
% has grown to a considerable length, she starts mining $m$ blocks starting from
% the Genesis block. After having mined $m$ blocks, the adversary produces a
% succinct proof of $m$ blocks at level $0$ and sends it to the verifier. The
% honest parties produce a succinct proof of higher level as usual. The verifier,
% comparing the two proofs, observes that the lowest common ancestor of the two
% proofs is the Genesis block. Hence, it requests the honest chain of level $0$
% starting from Genesis. However, this is nothing but the whole blockchain,
% requiring the honest parties to send a large amount of data.
%
% Interestingly, this attack is an amplification attack: While the adversary is
% only required to send a small amount of data, the honest parties are forced to
% send large amounts of data. Furthermore, observe that the adversary does not
% even have to mine the first $m$ blocks; instead, they can produce the first $m$
% honest blocks of the blockchain. If the highest superblock level of at least
% $m$ blocks of the honest chain is high enough, these $m$ blocks will not have a
% common block with the honest proof.
