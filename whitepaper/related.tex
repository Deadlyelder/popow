\section{Related work}

\paragraph{Lightweight ``SPV'' Clients.}
Nakamoto's original Bitcoin whitepaper~\cite{bitcoin} anticipated the rising costs of an evergrowing blockchain, and proposed a protocol for lightweight clients, called ``Simplified Payment Verification'' (SPV). Unlike ``full nodes'' which process and validate the entire blockchain (including all transactions and signatures), SPV clients only process the proof-of-work chain and transactions directly pertaining to them.
When initialized for the first time, an SPV client downloads (and stores) the entire chain blockchain, which is thus linear in $|\chain|$, and therefore grows unbounded over time, although it has low constant fators, i.e.
$80|\chain|$ bytes.
When an SPV client receives a payment, it requests a proof-of-inclusion (i.e., a Merkle tree branch proof) in order to confirm that the transaction is included in one of the blocks whose header is already stored by the client.
Most widely used clients, especially mobile wallets based on BitcoinJ, implement SPV. 

%It is safe for users to run SPV clients; they enjoy the same security guarantees as if they run full nodes, as long as we assume a majority of miners faithfully process all transactions. However, in July 2015, a miners were following not fully verifying transactions, 

As an alternative to downloading all block headers at first startup, the SPV client software could embed a hardcoded checkpoint (perhaps chosen conservatively, e.g. many months in the past), blocks prior to which are ignored.\footnote{See \url{https://bitcoinj.github.io/speeding-up-chain-sync}} The BitcoinJ software provides mechanisms for users to build ``Checkpoint Files,'' and at least some applciations, such as Multibit,\footnote{\url{https://multibit.org/help/hd0.3/how-spv-works.html}} includes checkpoints. Although this approach is very efficient, it introduces additional trust assumptions on software maintainers. 
\anote{THIS IS BROKEN BECAUSE MULTIBIT SEEMS TO EMBED CHECKPOINTS}
%~\cite{betterspv}, clients at first startup could proceed optimistically,

\paragraph{Proofs of Proofs of Work}
Kiayias et al. (KLS)~\cite{KLS} proposed a variation of the SPV protocol that avoids the need to transmit the entire chain of proof-of-work. Instead, in their scheme only a sublinear-sized sample of the proofs-of-work are sent, i.e. a ``proof of proof-of-work'' or PoPoW. They define security in a setting with a verifier (modeling the client) and multiple provers (modeling peers in the network), at least one of which is assumed to be honest. The KLS protocol is interactive; the verifier may need to interact with the provers over a number of rounds logarithmicin the length of the proof-of-work chain.
Our protocol is inspired by KLS, but improves on it in several ways. First, our protocol is non-interactive, completing in one communication round regardless of adversarial prover's behavior. We also provide a concrete implementation including practical optimizations to reduce proof size.

In a highly influential whitepaper~\cite{sidechains}, Back et al. proposed the use of efficient non-interactive SPV proofs as a way of enabling applications spanning multiple cryptocurrencies, called. One such application would be a virtual asset in one cryptocurrency, called a ``sidechain,'' that is backed by a deposit of value in another cryptocurrency. Several subsequent efforts by Poelstra~\cite{pos} and Friedenbach~\cite{compactspv} explored this design space, however a security definition and suitable construction has remained an open problem. In particular, prior proposals have been vulnerable to ``high variance'' attacks, i.e. attacks that fail on average, but nonetheless succeed with non-negligible probability. In Section~\ref{sec:applications}, we explain how our protocol solves this open problem and can fulfill the sidechains vision.

% \Section{Interactive Stateless Verification}
%
% The Nakamoto protocol is modified as such:
% - Every node still works to extend the longest valid chain.
%
% A ``score'' is associated to, based on the number of leading 0-bits in the
% ``score'' label. Thus $X = score(n)$ is distributed according to a geometric distribution (with parameter 0.5).
%
% Validity is modified as such:
% - Each node contains several links. One is to the most recent puzzle solution, similar to the original protocol.
% - Other links are to the most recent puzzle solution with $score(n) \geq
% 1$,  with $score(n) \geq 2$, and so on.
%
% Present a proof-of-proof-of-work $(X, \pi)$. $\pi^{\rceil k}$
%
% Disputing a false proof:
%     If the transaction in question is *not* in the longest valid chain, but the adversary presents a proof.
%
% Why is interaction needed?
%
% Links along the layer.
%
% \section{Compact SPV Proofs}
%
% Similar to Kiayias, we utilize backlinks, in effectively the same structure.
%
% Definitions:
% $(\alpha, k)$ - Segment proof for $[start, end]$
%
% Properties:
% with high probability, at least puzzle solutions that contain ``start'' as a descendant
%
% Efficiency:
% 	honest parties are able to produce a proof that contains fewer than $x$ blocks
%
% Prefix proof.  for [end]
%     The idea is to have successively larger proofs.
%
% Let $A$ be a valid puzzle chain.
%
% $A_{f}^{\mu}$, where $f \in A$, consists of the suffix of $A$ after
% $f$, such that $n \in A$, $score(n) > \mu$, and $f \subset_A n$.
%
% $\Pi_A[f]$  is defined by Alg 1.
%
% $\Pi_{A}$ is a pruned version of A, defined by Alg 1.
% $\Pi_{A}$ satisfies the following property:
%
% \begin{equation*}
%     \exists \mu \texttt{ such that } \Pi_{A}{f}^{\mu} = A_f^{\mu}
%     \texttt{ and }
%     |A_f^{\mu}| > k
%     \texttt{ or } \mu
% \end{equation*}
