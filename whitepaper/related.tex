\section{Related work}

\paragraph{Lightweight ``SPV'' Clients.}
Nakamoto's original Bitcoin whitepaper~\cite{bitcoin} anticipated the rising costs of an evergrowing blockchain, and proposed a protocol for lightweight clients, called ``Simplified Payment Verification'' (SPV). Unlike ``full nodes'' which process and validate the entire blockchain (including all transactions and signatures), SPV clients only process the proof-of-work chain and transactions directly pertaining to them.
When initialized for the first time, an SPV client downloads (and stores) the entire chain blockchain, which is thus linear in $|\chain|$, and therefore grows unbounded over time, although it has low constant fators, i.e.
$80|\chain|$ bytes.
When an SPV client receives a payment, it requests a proof-of-inclusion (i.e., a Merkle tree branch proof) in order to confirm that the transaction is included in one of the blocks whose header is already stored by the client.
Most widely used clients, especially mobile wallets based on BitcoinJ, implement SPV.

%It is safe for users to run SPV clients; they enjoy the same security guarantees as if they run full nodes, as long as we assume a majority of miners faithfully process all transactions. However, in July 2015, a miners were following not fully verifying transactions, 

As an alternative to downloading all block headers at first startup, the SPV client software could in principle embed a hardcoded checkpoint (perhaps chosen conservatively, e.g. many months in the past) blocks prior to which the client ignores. This would introduce additional assumptions about the maintenance of the code; although BitcoinJ software provides a mechanism for users to build ``Checkpoint Files,''\footnote{https://bitcoinj.github.io/speeding-up-chain-sync} but does not embed these. Similarly, ~\cite{betterspv}, clients at first startup could proceed optimistically,
\anote{THIS IS BROKEN BECAUSE MULTIBIT SEEMS TO EMBED CHECKPOINTS}

\paragraph{BTCRelay.}
BTCRelay~\cite{btcrelay} is an Ethereum~\cite{ethereum} smart contract written in Ethereum that acts as an SPV client for the Bitcoin blockchain.
That is, the smart contract contains rules for parsing 
This allows Ethereum smart contract to condition their behavior based on the presence of particular transactions in the Bitcoin blockchain.
Users can interact with the client to submit Bitcoin.

At the current time, Ethereum does not provide built-in support for other hashes like scrypt, or even EtHash, the basis for Ethereum's own proof of work. Ethereum could be upgraded to support such opcodes in the future. This would 

\paragraph{Cross Chain Transactions.}
An application of our technique is for sidechains.
Hash lock contract are supported by most cryptocurrency scripting languages, including Bitcoin script and the Ethereum virtual machine. Only relies on the ability to verify that a user provided input is the preimage of a given hash.

In \cite{KLS}, it was discovered that these proofs can be sublinear in size. In
that paper they achieved proof sizes constant with respect to the chain size
$\Theta(m + k)$, but these proofs required interactivity. Furthermore, the
succinctness argument only worked for the optimistic case where all parties
were honest.

Improving on both of these works, we successfully presented PoPoWs that are
both sublinear and non-interactive.

Efficiency of PoPoWs has been explored in the blockchain community especially
around the topic of sidechains \cite{sidechains}, but to our knowledge no
concrete provably secure solution has been previously proposed.

% \section{Interactive Stateless Verification}
%
% The Nakamoto protocol is modified as such:
% - Every node still works to extend the longest valid chain.
%
% A ``score'' is associated to, based on the number of leading 0-bits in the
% ``score'' label. Thus $X = score(n)$ is distributed according to a geometric distribution (with parameter 0.5).
%
% Validity is modified as such:
% - Each node contains several links. One is to the most recent puzzle solution, similar to the original protocol.
% - Other links are to the most recent puzzle solution with $score(n) \geq
% 1$,  with $score(n) \geq 2$, and so on.
%
% Present a proof-of-proof-of-work $(X, \pi)$. $\pi^{\rceil k}$
%
% Disputing a false proof:
%     If the transaction in question is *not* in the longest valid chain, but the adversary presents a proof.
%
% Why is interaction needed?
%
% Links along the layer.
%
% \section{Compact SPV Proofs}
%
% Similar to Kiayias, we utilize backlinks, in effectively the same structure.
%
% Definitions:
% $(\alpha, k)$ - Segment proof for $[start, end]$
%
% Properties:
% with high probability, at least puzzle solutions that contain ``start'' as a descendant
%
% Efficiency:
% 	honest parties are able to produce a proof that contains fewer than $x$ blocks
%
% Prefix proof.  for [end]
%     The idea is to have successively larger proofs.
%
% Let $A$ be a valid puzzle chain.
%
% $A_{f}^{\mu}$, where $f \in A$, consists of the suffix of $A$ after
% $f$, such that $n \in A$, $score(n) > \mu$, and $f \subset_A n$.
%
% $\Pi_A[f]$  is defined by Alg 1.
%
% $\Pi_{A}$ is a pruned version of A, defined by Alg 1.
% $\Pi_{A}$ satisfies the following property:
%
% \begin{equation*}
%     \exists \mu \texttt{ such that } \Pi_{A}{f}^{\mu} = A_f^{\mu}
%     \texttt{ and }
%     |A_f^{\mu}| > k
%     \texttt{ or } \mu
% \end{equation*}
