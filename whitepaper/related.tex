\section{Related work}

In the paper which introduced bitcoin \cite{bitcoin}, Nakamoto was well aware
of the requirement for shorter proofs-of-work. He proposed SPV clients there,
which, while linear in $|\chain|$ had low coefficients and achieved proofs of
$80|\chain|$ bytes, given that each block header is $|\chain|$ long (in
addition to the size required by the Merkle proofs-of-inclusion).

In \cite{KLS}, it was discovered that these proofs can be sublinear in size. In
that paper they achieved proof sizes constant with respect to the chain size
$\Theta(m + k)$, but these proofs required interactivity. Furthermore, the
succinctness argument only worked for the optimistic case where all parties
were honest.

Improving on both of these works, we successfully presented PoPoWs that are
both sublinear and non-interactive.

Efficiency of PoPoWs has been explored in the blockchain community especially
around the topic of sidechains \cite{sidechains}, but to our knowledge no
concrete provably secure solution has been proposed.

% \section{Interactive Stateless Verification}
%
% The Nakamoto protocol is modified as such:
% - Every node still works to extend the longest valid chain.
%
% A ``score'' is associated to, based on the number of leading 0-bits in the
% ``score'' label. Thus $X = score(n)$ is distributed according to a geometric distribution (with parameter 0.5).
%
% Validity is modified as such:
% - Each node contains several links. One is to the most recent puzzle solution, similar to the original protocol.
% - Other links are to the most recent puzzle solution with $score(n) \geq
% 1$,  with $score(n) \geq 2$, and so on.
%
% Present a proof-of-proof-of-work $(X, \pi)$. $\pi^{\rceil k}$
%
% Disputing a false proof:
%     If the transaction in question is *not* in the longest valid chain, but the adversary presents a proof.
%
% Why is interaction needed?
%
% Links along the layer.
%
% \section{Compact SPV Proofs}
%
% Similar to Kiayias, we utilize backlinks, in effectively the same structure.
%
% Definitions:
% $(\alpha, k)$ - Segment proof for $[start, end]$
%
% Properties:
% with high probability, at least puzzle solutions that contain ``start'' as a descendant
%
% Efficiency:
% 	honest parties are able to produce a proof that contains fewer than $x$ blocks
%
% Prefix proof.  for [end]
%     The idea is to have successively larger proofs.
%
% Let $A$ be a valid puzzle chain.
%
% $A_{f}^{\mu}$, where $f \in A$, consists of the suffix of $A$ after
% $f$, such that $n \in A$, $score(n) > \mu$, and $f \subset_A n$.
%
% $\Pi_A[f]$  is defined by Alg 1.
%
% $\Pi_{A}$ is a pruned version of A, defined by Alg 1.
% $\Pi_{A}$ satisfies the following property:
%
% \begin{equation*}
%     \exists \mu \texttt{ such that } \Pi_{A}{f}^{\mu} = A_f^{\mu}
%     \texttt{ and }
%     |A_f^{\mu}| > k
%     \texttt{ or } \mu
% \end{equation*}
