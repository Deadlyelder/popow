\section{Model}

We work on the Backbone model \cite{backbone}. We introduce two entities to
the standard Backbone model: The Prover and the Verifier which work in the
existing environment. The Prover is part of the existing miner node which
exists in the backbone protocol. The Verifier is a completely new entity. When
the game begins, the Prover and the Verifier are instantiated with a fixed
predicate $Q$, which generalized the notion of proof-of-mining and allows any
blockchain property to be proven between the Prover and the Verifier. Some of
these predicates are more suitable for succinct proofs than others. We focus
our attention in what we call \textit{ reliable} predicates; that is,
predicates which are monotonic and stable.  These predicates have the
exceptional property that all honest miners share their view of them in a way
that is updated in a predictable manner, with a decision that persists as the
blockchain grows.  Furthermore, we are interested in \textit{succinct}
predicates, predicates which can be proven using a succinct protocol, a
protocol which requires proofs that are short in terms of the total blockchain
size. Finally, we formalize the notion of non-interactivity.

We use this model to explore which predicates are reliable and can be proven
succinctly.

\import{./}{algorithms/alg.backbone.tex}
\import{./}{algorithms/alg.backbone-prover.tex}

For blockchain addressing we use the Python range notation of brackets ($[]$).
A positive number in a bracket indicates the indexed block in the blockchain,
starting from zero. A negative number indicates a block indexed from the end. A
range of two numbers separated by a $:$ indicates a subchain starting on the
left index and ending on the right index, inclusive. Omitting one of the range
indices takes it to the beginning or end of the blockchain respectively.
