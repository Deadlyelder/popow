\section{Model}

We define our problem in the setting of a blockchain client which is completely
stateless beyond the knowledge of the Genesis block. Without loss of
generality, this single reference block could be any stable checkpoint block,
either hard-coded in the client or obtained through a previous interaction with
the network. However, importantly, the client does not maintain a blockchain.
The challenge is, then, to build a client that is able to communicate with the
network and, without downloading the whole chain headers, is convinced in a
secure manner that a transaction has taken place. We term such a client a
\textit{verifier} and we term \textit{provers} the full nodes it connects to.
Following previous work \cite{backbone} \cite{KLS}, we assume the verifier will
always connect to at least one honest prover. We now move on to formally define
these entities in terms of a mathematical model which can be used for our
analysis.

In order to study the syntax and properties of NIPoPoWs
we abstract their operation in a way that is akin to non-interactive
zero-knowledge proofs (NIZK) \cite{BFM88}. In such protocols, a prover
shares some context with a verifier (called the common reference string).
For a given predicate, the prover computes the proof string and transmits it
to the verifier. The verifier accepts or rejects the string. Given that in our
setting the predicates will be referring to an active evolving chain 
we should define them in the presence of a blockchain protocol. 
Our theoretical abstraction is based on ``backbone'' model of
\cite{backbone} that formalizes the operation of the bitcoin protocol.
We introduce the Prover and the Verifier in
the standard Backbone model as follows. The Prover will extend the functionality
of the miner node. The Verifier will be a completely new entity that can be spawned
by the environment and process a NiPoPow produced by the miner nodes running
the Prover code. 

The predicates of interest in our context are predicates on the active blockchain. 
Some of the predicates are more suitable for (succinct) proofs than others. We focus
our attention in what we call \textit{reliable} predicates; that is,
predicates which are monotonic and stable notions that we define below.  
These predicates have the
exceptional property that all honest miners share their view of them in a way
that is updated in a predictable manner, with a truth-value that persists as the
blockchain grows.  Furthermore, we are interested in \textit{succinct}
predicates, predicates which can be proven using a succinct protocol, a
protocol which requires proofs that are short in terms of the total blockchain
size. Finally, we formalize the notion of non-interactivity.

%We use this model to explore which predicates are reliable and can be proven
%succinctly.

\import{./}{algorithms/alg.backbone.tex}
\import{./}{algorithms/alg.backbone-prover.tex}
\import{./}{algorithms/alg.verifier-framework.tex}

For blockchain addressing we use the Python range notation of brackets ($[]$).
A positive number in a bracket indicates the indexed block in the blockchain,
starting from zero. A negative number indicates a block indexed from the end. A
range of two numbers separated by a $:$ indicates a subarray starting on the
left index and ending on the right index, inclusive. Omitting one of the range
indices takes it to the beginning or end of the blockchain respectively.

The backbone protocol of a miner node is shown in Algorithm~\ref{alg.backbone}.
Its rationale is described in more detail in \cite{backbone}. In short, the
honest miner maintains the longest chain from the network and tries to mine on
top of it. In Algorithm~\ref{alg.backbone-prover}, we illustrate our new entity
for the model, the full node or \textit{prover}. While the full node is not
mining, it maintains a state with the longest chain from the network.
Furthermore, whenever it is asked to prove a predicate $Q$ about its local
chain $\chain$, it calls a Prove function to provide the proof. We leave this
Prove function undefined for now, as it is part of the concrete protocol
construction. On the other hand, in Algorithm~\ref{alg.generic-verifier} we
illustrate another new entity to the model, the generic \textit{verifier},
which is stateless, but receives proofs from the provers on the network (via
the environment) and takes a decision about a predicate on the blockchain it
believes to be the longest.

A non-interactive proof of proof-of-work protocol is a game between a prover
and a verifier parameterized by predicate $Q$ which the prover tries to
convince the verifier for. The predicate $Q$ is a function of a chain $\chain$,
which the prover claims to be the longest chain.

At the beginning of the game, two provers generate proofs $\pi_A$, $\pi_B$
claiming potentially different truth values for the predicate $Q$ based on
their claimed local longest chains. The verifier receives these proofs and
accepts one of the two proofs, determining the truth value of the predicate.

We define a \textit{blockchain proof protocol} for a predicate $Q$ as a pair
$(P, V)$ where $P$ is called the \textit{prover} and $V$ is called the
\textit{verifier}. $P$ is a PPT algorithm that is spawned by a full node
when they wish to produce a proof, accepts as input a full chain $\chain$ and
produces a proof $\pi$ as its output. $V$ is a PPT algorithm which is spawned
by the environment at some round, receives a vector of proofs $\bar{\pi}$ from
both honest parties and the adversary and returns its decision $d \in \{T, F,
\bot\}$ before the next round and terminates. The honest parties produce proofs
for $V$ using $P$.
