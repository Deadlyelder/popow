\section{Model and Definitions}
\label{sec.model}

We define our problem in the setting of a blockchain client which is completely
stateless beyond the knowledge of a common reference string, the Genesis block.
Without loss of generality, this single reference block could be any stable
checkpoint block, either hard-coded in the client or obtained through a previous
interaction with the network. However, importantly, the client does not maintain
a blockchain. The challenge is, then, to build a client that is able to
communicate with the network and, without downloading the whole chain headers,
is convinced in a secure manner that a transaction has taken place. We term such
a client a \textit{verifier} and we term \textit{provers} the full nodes it
connects to.

The intuition behind these proofs is quite simple: Let $T$ be the mining target
for the network. Then because block ids are generated uniformly at random, half
of the blocks will have an id less than $T / 2$. By having the prover present to
the verifier just these blocks which capture more proof-of-work than their
neighbor counterparts, the verifier is convinced that the proof-of-work of the
omitted blocks also took place. The same logic can be applied to any level
$\mu$, namely blocks with id less than $T 2^{-\mu}$.

Before we present our construction in detail, we lay out the mathematical model
in which we will be working in this section.

\subsection{Overview of the backbone model}

In order to make the analysis mathematically precise, we adopt the backbone
model of GKL~\cite{GKL} which formally captures the notion of an evolving
blockchain. This subsection provides an introduction to that model.

The backbone model is parameterized by the parameters $\kappa, n, t, p, q$. The
meaning of these parameters will become clear shortly.

Backbone models the blockchain as an execution of a single Interactive Turing
Machine (ITM), the \textit{environment}, which is responsible for coordinating
the game. There are two more ITM specifications: The honest party  and the
adversary. The environment is responsible for spawning them and scheduling them
for execution.

Two parameters of the protocol are $n$, the total number of parties, and $t$ the
number of adversarial parties. Therefore, $n - t$ is the number of honest
parties. At the beginning of the game, the environment spawns $n - t$ honest
parties which are termed $\Pi_1, \Pi_2, \cdots, \Pi_{n - t}$ and exactly one
adversary $\mathcal{A}$. The environment then generates a common reference
string, the \textit{Genesis} block, which is provided to all of these players by
invoking a certain initialization subroutine.

The execution of the game happens in discrete \textit{rounds} which are used to
model synchronous communication between the various players. The environment
runs a main loop and maintains the current round count. Each iteration of this
loop constitutes one round. At the beginning of each round, a \textit{round
execution subroutine} defined by $\Pi_1$ is invoked. When $\Pi_1$'s subroutine
is completed, the subroutine is invoked for $\Pi_2$ and so on until $\Pi_{n -
t}$ finishes executing their subroutine. In each of these executions, the player
$\Pi_i$ being invoked is not aware of their index $i$, but can maintain local
state which persists across rounds. At the end of each round, the round
execution subroutine of the adversary is executed. The adversary can also
maintain local state across rounds. The maintenance of local state is formally
modelled using Interactive Turing Machines. The fact that each honest party is
unaware of its own index \dznote{XXX finish this section}.

The environment also spawns a random oracle functionality $\mathcal{H}$ which
produces truly random responses in the range $\{0, 1\}^\kappa$ to queries.

\subsection{New algorithms in the backbone model}

% \dznote{Rewrite this section more extensively}

We assume the verifier will always connect to at least one honest prover. We
now move on to formally define these entities in terms of a mathematical model
which can be used for our analysis.

In order to study the syntax and properties of NIPoPoWs
we abstract their operation in a way that is akin to non-interactive
zero-knowledge proofs (NIZK) \cite{BFM88}. In such protocols, a prover
shares some context with a verifier (called the common reference string).
For a given predicate, the prover computes the proof string and transmits it
to the verifier. The verifier accepts or rejects the string. Given that in our
setting the predicates will be referring to an active evolving chain
we should define them in the presence of a blockchain protocol.
Our theoretical abstraction is based on  the ``backbone'' model of
\cite{backbone} that formalizes the operation of the Bitcoin protocol.

The backbone approach models proof-of-work discovery attempts by using a random
oracle~\cite{RO}. The model uses previous formulations of secure multiparty
computation, specifically Canetti's formulation of ``real world'' execution
~\cite{uc0}~\cite{uc1}~\cite{uc2}. For following the full formal security
analyses, we kindly invite our reader to familiarize themselves with the
backbone model of GKL\cite{backbone} and the respective quantities.

We introduce the Prover and the Verifier in
the Backbone model as follows. The Prover will extend the functionality
of the miner node. The Verifier will be a completely new entity that can be spawned
by the environment and process a NiPoPoW produced by the miner nodes running
the Prover code.

% Detailed algorithms for the integration with the formal
% environment provided by the backbone paper are provided in
% Appendix~\ref{sec.appendix-backbone}.

The predicates of interest in our context are predicates on the active
blockchain. Some of the predicates are more suitable for succinct proofs than
others. We focus our attention in what we call \textit{reliable} predicates;
that is, predicates which are monotonic and stable notions that we define below.
These predicates have the exceptional property that all honest miners share
their view of them in a way that is updated in a predictable manner, with a
truth-value that persists as the blockchain grows.  Furthermore, we are
interested in \textit{succinct} predicates, predicates which can be proven using
a succinct protocol, a protocol which requires proofs that are short in terms of
the total blockchain size. Finally, we formalize the notion of non-interactivity.

The honest entities that live on the blockchain network are of three kinds:
First, miners, who try to mine new blocks on top of the longest known blockchain
and broadcast their blocks as soon as they are discovered. Second, full nodes,
which maintain the longest blockchain but without mining. Full nodes also act as
the provers in the network, producing proofs by calling a Prove function which
we leave undefined for now, as it is part of the proof protocol. And third,
verifiers or stateless clients, which connect to provers and ask for proofs in
regards to which blockchain is the largest. The verifiers attempt to determine the value of a predicate on these chains.

A non-interactive proof of proof-of-work protocol is a game between a prover
and a verifier parameterized by predicate $Q$ which the prover tries to
convince the verifier for. The predicate $Q$ is a function of a chain $\chain$,
which the prover claims to be the longest chain.

When asked to do so by the environment, two provers generate proofs $\pi_A$,
$\pi_B$ claiming potentially different truth values for the predicate $Q$ based
on their claimed local longest chains. The verifier receives these proofs and
accepts one of the two proofs, determining the truth value of the predicate. In
this setting, one of the two proofs could be adversarial, as defined in the
security section.

We define a \textit{blockchain proof protocol} for a predicate $Q$ as a pair
$(P, V)$ where $P$ is called the \textit{prover} and $V$ is called the
\textit{verifier}. $P$ is a PPT algorithm that is spawned by a full node
when they wish to produce a proof, accepts as input a full chain $\chain$ and
produces a proof $\pi$ as its output. $V$ is a PPT algorithm which is spawned
by the environment at some round, receives a pair of proofs $(\pi_A, \pi_B)$
from both an honest party and the adversary and returns its decision $d \in \{T,
F, \bot\}$ before the next round and terminates. The honest parties produce
proofs for $V$ using $P$.

\subsection{Blockchain addressing}

Blockchains, or \textit{chains}, are finite sequences of blocks such that they
obey the \textit{blockchain property} that for every block in the chain there
exists a pointer to its previous block in the chain. A chain $\chain$ is
\textit{anchored} if its first block is the designated \textit{Genesis} block, a
known reference string. We denote the genesis block as $Gen$.

For chain addressing we use the Python range notation of brackets $[~]$. A
positive number in a bracket indicates the indexed block in the chain, starting
from zero. For example, if $\chain$ is anchored, then $\chain[0] = Gen$. A
negative number indicates a block indexed from the end.  For example
$\chain[-1]$ indicates the tip of the chain. A range of two numbers
$\chain[i:j]$ indicates a subarray starting on index $i$ (inclusive) and ending
on index j (exclusive), which can be negative. Omitting one of the range indices
takes it to the beginning or end of the chain respectively. So, for example
$\chain[1:]$ indicates a chain with the first block omitted and $\chain[:-k]$
indicates a chain with the last $k$ blocks omitted, the ``stable'' part of the
chain according to \cite{backbone}.

Given two chains $\chain_1$ and $\chain_2$ we write $\chain_1 \chain_2$ to
indicate the concatenation of these two chains into a new chain. Then the
$\chain_2[0]$ must point back to $\chain_1[-1]$. Similarly with a block $B$ we
write $\chain_1 B$ to indicate the concatenation of block $B$ to the tip of
$\chain_1$, implying that block $B$ points back to $\chain_1[-1]$ to preserve
the blockchain property.

We will also use block-based addressing with the range notation of curly braces $\{~\}$.
Given a chain $\chain$ and two blocks $A, Z$ which are in $\chain$, we denote
$\chain\{A:Z\}$ the subarray of the chain starting from block $A$ (inclusive)
and ending in block $Z$ (exclusive). Again we can omit blocks from the range to
take the chain to the beginning or end respectively. Hence, for example,
$\chain\{A:\}$ indicates that the subarray of the chain from block $A$ and until
the end of the chain is to be taken.

Some helper functions need to be defined. The \textit{id} function
calculates the id of a block given its data. This is done by applying the
hashfunctions $H$ and $G$ in a manner similar to \cite{backbone}: $\textsf{id} =
H(ctr, G(s, x, \textsf{interlink}))$. However, note that block $B$ maintains an
\textit{interlink} data structure instead of a previous block pointer. We will
explain this further in section \ref{sec.consensus}. Note also that, while we
spell out the transaction vector $x$, this in practice contains only the root of
the Merkle tree of the transactions \cite{princetonbook}.

The id of a block must always be less than or equal to the mining target $T$ to
satisfy the proof-of-work\cite{antonopoulosbook} condition: $id \leq T$.
However, some blocks will achieve a lower id. If $id \leq \frac{T}{2^\mu}$ we
say that the block is of level $\mu$. Clearly all blocks are blocks of level
$0$. We term blocks that have a level above 0 \textit{superblocks} of their
respective level. The level of a block is given as $\mu = \left \lfloor \log(T) -
\log(\sf{id}(B)) \right \rfloor$ and denoted $\textit{level}(B)$. By convention,
we set the id of Genesis to be $0$ and hence it has $\mu = \infty$.

In this paper, we will extend blocks to contain multiple pointers to previous
blocks. Hence, certain blocks can be omitted from a chain, obtaining a subchain.
However, the subchain must still maintain the blockchain property that each
block must contain a pointer to its previous block in the sequence (except for
the first block in the sequence). Subarrays of chains obtained by addressing
with $[~]$ or $\{~\}$ are, naturally, subchains, as the only omit blocks from
the beginning or the end of a chain. It will also become apparent that it is
useful to omit certain blocks from the middle of a chain, while maintaining the
blockchain property.

While chains are sequences and not sets, we will use some set notation for
convenience. The notation $B \in \chain$ will be used to indicate that a block
$B$ is part of a chain $\chain$, while $B \not\in \chain$ will be used to
indicate that it isn't. We will use $\emptyset$ to denote the empty blockchain.
Finally, given a chain $\chain$ and two subchains $\chain_1$ and $\chain_2$ of
it, we will write $\chain_1 \subseteq \chain_2$ to indicate that $\chain_1$ is a
(potentially improper) subchain of $\chain_2$, i.e. that for all blocks $B$ of
$\chain_1$, we have that $B \in \chain_2$, while $\chain_1 \not\subseteq
\chain_2$ will be used to indicate that it isn't. Because both $\chain_1
\subseteq \chain$ and $\chain_2 \subseteq \chain$, we can define $\chain_1 \cup
\chain_2$ to mean the chain obtained by sorting the blocks contained in both
$\chain_1$ and $\chain_2$ into a sequence, provided that we can ensure the
blockchain property is maintained. Hence, for example, if blocks $(B_1, B_2,
B_3)$ form a blockchain $\chain$ such that $B_2$ points to $B_1$ and $B_3$
points to $B_2$, but also $B_3$ points to $B_1$, then letting $\chain_1 = (B_1,
B_3)$ and $\chain_2 = (B_2)$, we note that both $\chain_1$ and $\chain_2$ are
blockchains and that $\chain_1 \cup \chain_2$ is a blockchain. Additionally,
take two anchored chains $\chain_1$ and $\chain_2$, and note that $\chain_1$ and
$\chain_2$ belong to the same blocktree but can potentially have forked. Then we
use the notation $\chain_1 \cap \chain_2$ to mean the anchored chain $\{B: B \in
\chain_1 \land B \in \chain_2\}$ provided it satisfies the blockchain property.
We also define the lowest common ancestor of these chains $\textsf{LCA}(\chain_1,
\chain_2) = (\chain_1 \cap \chain_2)[-1]$. We will also freely use set builder
notation $\{B \in \chain: p(B)\}$ to obtain a subchain $\chain'$ from a chain
$\chain$ by filtering with a block predicate $p(B)$, provided the blockchain
property is maintained. Note that in all these cases, because we are talking
about blockchains and not blocktrees, any set of blocks has a unique order
obtained by topologically sorting the blocks based on the order indicated by the
underlying chain and so can be put into a unique sequence. For two chains
$\chain_1$ and $\chain_2$, if $\chain_1[0] = \chain_2[0]$ and $\chain_1[-1] =
\chain_2[-1]$, we will say that the two chains \textit{span} the same range of
blocks.

Given a chain $\chain$ and a level $\mu$, we define the \textit{upchain
operator} $\chain\upchain^\mu$ to mean the chain $\{B \in \chain: level(B) \geq
\mu\}$. We will call such a chain that contains only $\mu$-superblocks a
$\mu$\textit{-superchain}. Given a chain $\chain'$ and a chain $\chain$ such
that $\chain' \subseteq \chain$, we define the \textit{downchain operator}
$\chain'\downchain^\chain$ as the chain $\chain[\chain'[0]:\chain'[-1]]$ and we
will call $\chain$ the \textit{underlying chain} of $\chain'$. If the underlying
chain is implied by context, for example, if the chain $\chain'$ was obtained by
applying the upchain operator on a known underlying chain $\chain$, we will
simply write $\chain'\downchain$. Note as well that the $\chain\upchain$
operator is absolute, in the sense that $(\chain\upchain^\mu)^{\mu + i} =
\chain\upchain^{\mu + i}$.

If a set of consecutive rounds $S$ is given, we define \[\chain^S = \{B \in
\chain: B \text{ was generated during } S\}\].
