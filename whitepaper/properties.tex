\subsection{Desired properties}

We now define two desired properties of a non-interactive proof protocol, in
particular \textit{succinctness} and \textit{security}.

\begin{definition}{(Security)}
A \textit{blockchain proof protocol} is \textit{secure} if for all environments
and for all PPT adversaries $\mathcal{A}$ the output of $V$ on round $r$ is the
same as the evaluation of $Q(\chain)$ on all honest parties' chains $\chain$ as
long as $Q(\chain)$ is the same for all honest parties after round $r$.
\end{definition}

\begin{definition}{(Succinctness)}
A \textit{blockchain proof protocol} is \textit{succinct} if the maximum proof
size $|\pi|$ across all honest provers for a given round $r$ is
$O(polylog(r))$.
\end{definition}

Based on these definitions, it is easy to see that it is trivial to construct a
secure but not succinct protocol: The prover provides the chain $\chain$ itself
as a proof and the verifier simply compares the proofs received by length:
Since the longest proof is literally the longest chain, the protocol is
trivially secure but not succinct. The challenge we will solve over the next
sections is to provide a non-interactive protocol which at the same time
achieves security and succinctness over a large class of predicates.

% we now provide the model for a stateless typical
% bitcoin SPV node in Algorithm~\ref{alg.verifier-full}. This verifier is
% trivially secure, but not succinct.

% \import{./}{algorithms/alg.verifier-full.tex}
