\section{Desired properties of blockchain proofs}

We now move on to define the various desired properties of a proof protocol, in
particular \textit{succinctness} and \textit{security}.

We define a \textit{blockchain proof protocol} for a predicate $Q$ as a pair
$(P, V)$ where $P$ is called the \textit{prover} and $V$ is called the
\textit{verifier}. $P$ is a PPT algorithm that is spawned by an honest miner
when they wish to produce a proof, accepts as input a full chain $\chain$ and
produces a proof $\pi$ as its output. $V$ is a PPT algorithm which is spawned
by the environment at some round, receives a vector of proofs $\bar{\pi}$ from
both honest parties and the adversary and returns its decision $d \in \{T, F,
\bot\}$ before the next round and terminates. The honest parties produce proofs
for $V$ using $P$.

\begin{definition}{(Security)}
A \textit{blockchain proof protocol} is \textit{secure} if for all environments
and for all PPT adversaries $\mathcal{A}$ the output of $V$ on round $r$ is the
same as the evaluation of $Q(\chain)$ on all honest parties' chains $\chain$ as
long as $Q(\chain)$ is the same for all honest parties after round $r$.
\end{definition}

\begin{definition}{(Succinctness)}
A \textit{blockchain proof protocol} is \textit{succinct} if the maximum proof
size $|\pi|$ across all honest provers for a given round $r$ is
$O(polylog(r))$.
\end{definition}

Based on these definitions, we now provide the model for a stateless typical
bitcoin SPV node in Algorithm~\ref{alg.verifier-full}. This verifier is
trivially secure, but not succinct.

\import{./}{algorithms/alg.verifier-full.tex}
