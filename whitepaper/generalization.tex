\section{Infix proofs}

\subsection{Construction}

While the previous proofs we constructed were proving predicates on the blocks
and transactions pertaining to the chain suffix of the last $k$ blocks, similar
to the kind of proofs possible with \cite{KLS}, we note that the most useful
class of proofs would allow proving more general predicates that can depend on
multiple blocks, sometimes even blocks burried deep within the blockchain.

We will now extend our prover to support this generalization of predicates. The
generalized prover allows proving any predicate $Q(\chain)$ that depends on a
number of blocks that can appear anywhere within the chain. These blocks
constitute a \textit{subchain} $\chain'$ which is a subsequence of the original
blockchain. For our proofs to work, the blocks contained within the subchain
have to be $k$-stable; that is, they cannot be one of the most recent $k$
blocks.

The proofs are able to prove any predicate of the general class of predicates
that depend on both the inclusion of the blocks of the subchain as well as
their relative order within the subchain. This allow proving powerful
statements such as, for example, whether a transaction took place at any point
in history, or even ``following the money'' and proving a claim that a certain
coin was moved between a series of addresses in a particular order.

\import{./}{algorithms/alg.nipopow-infix-follow.tex}
\import{./}{algorithms/alg.nipopow-infix-prover.tex}

\subsection{Security}
The security of infix proofs follows immediately from
Theorem~\ref{thm.security}. To see this, notice that the verifier will still
verify the suffix proof. Upon determining which of the blockchains is the
longest securely, it will subsequently verify that the subchain blocks have
been properly included within the referenced chain. However, if the chain is
honest, this proof of inclusion cannot be produced by the adversary.

\subsection{Succinctness}
As long as the number of blocks on which the predicate depends is
polylogarithmic with respect to the chain length, our proofs remain succinct.
% Specifically, the proof size for the suffix has exactly the same size. Then
% TODO
