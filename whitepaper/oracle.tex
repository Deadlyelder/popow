\section{Simplified mining model}

In order to be precise, while avoiding having to wave away network details and
message formats complicate our analysis, we describe in terms of interaction
with an oracle.

Instead of performing work (attempting to compute values with special images
under a hash function), simply interact with the oracle.

Intuitively, the oracle keeps track of the various puzzle graphs computed by
various parties, including the adversary.

\section{Puzzle Graph Oracle}

Puzzle graphs are directed graphs consisting of nodes, called puzzle solutions.

A puzzle solution is ``public'' or ``private''. Honest parties are able to see
the graph consisting of all the ``public'' puzzle solutions, and the (directed)
edges among them.

Puzzle solutions come with protocol-defined labels. The protocol may define
application semantics. For example, in our model the idea of ``included
transactions'' is a label. Our protocol will also include ``height'' as an
annotation. The protocol may define a ``validity'' condition.

The oracle interface is defined as follows:

    Graph is initially empty.

    on ReadPublicGraph():
        returns the graph consisting of all the public nodes and the (directed) edges between them

    on AddPuzzleSolution( label, edges ):
        Adds a new (public) node with the given label. Edges is a set of public nodes.
        A special label $\texttt{score}$ is chosen uniformly at random in the
        range $[0, 1]$.

    on AddPrivateSolution( label, edges ) from Adversary:
        same as AddPuzzleSolution, except the node is private. This may only be called by the adversary.


    on MakePublic( node ) from Adversary: change a node from private to public


    on PresentProof( recipient, nodes ):
        Presents a “pruned” graph, containing only nodes.
        Nodes is a subset of the nodes known to the oracle.
        The oracle sends to “recipient” the subgraph containing each node in $nodes$, all of the outgoing edge *labels* for each node.

Bounded rate oracle access:
    The number of oracle queries made by the adversary cannot exceed $f/N +
    O(1)$ of the number of oracle queries made by the honest parties.
