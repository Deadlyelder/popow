\section{Gradual Deployment Paths}
\label{sec:forks}
Our construction requires an upgrade to the cryptocurrency consensus layer.
We envision that new cryptocurrencies will adopt our construction in order to support efficient light clients.

However, existing cryptocurrencies could also benefit by adopting our construction as an upgrade.
In this section we outline several possible upgrade paths. We also contribute a novel upgrade approach, called a ``velvet fork,'' which allows for gradual deployment without harming unupgraded nodes or miners at all.

\subsection{Hard Forks and Soft Forks}

First of all, keep in mind that the interlink data structure will be useful to
be included in the form of a Merkle tree of interlink pointers, as discussed in
the next section on implementation optimizations.

The most obvious way to upgrade a cryptocurrency to support our protocol is a hard fork: the block header could be modified to include our additional interlink structure, and the validation rules would be modified to require that new blocks (after a ``flag day'') contain a correctly-formed interlink hash.

The safety of the hard fork approach is often debated: it is not ``forward compatible,'' in the sense that nodes failing to uprade in time would not be able to process newly created blocks. Hard forks are also associated with ``schisms'': in one notable case, Ethereum applied a hard fork in order to reverse the effects of an exploited vulnerability; this decision was controversial, and dissenters splintered off into a new network, Ethereum Classic~\cite{daofork}. However, both Ethereum and Ethereum Classic have subsequently applied hard fork upgrades without incident. The Monero cryptocurrency has applied hard forks on a predictable schedule~\cite{monerohardforks}.

  A soft fork construction in practice would require including the interlink data structure
not in the block header, but in the data of the coinbase transaction.  Clearly
it is enough to store a single hash of the whole interlink structure in the
coinbase data. The only requirement for the PoPoWs to work is that the
PoW commits to all the pointers within interlink so that the adversary cannot
cause a chain reorganization except with negligible probability. That way, each
superchain is really a chain which goes all the way back to genesis. If we take
that route, then each PoPoW will be required to present not only the block
header, but also a proof-of-inclusion path within the Merkle tree of
transactions proving that the coinbase transaction is indeed part of the block.
Once that is established, the coinbase data can be presented, and the verifier
will thereby know that the hash of the interlink data structure is correct.
Given the fact that in the current bitcoin implementation there is a fixed
limit for block sizes, we observe that including such proofs-of-inclusion will
only increase the PoPoW sizes by a constant factor per block, allowing for the
communication complexity to remain at $\Theta(polylog(|\chain|))$.


\subsection{Velvet Forks}
We now describe a novel upgrade path that avoids the need for a fork at all.
The key idea is that clients can make use of our scheme, even if only some blocks in the blockchain include the interlink structure.
Given that intuitively these
changes constitute rule modifications to the consensus layer, we call this
technique a \textit{velvet fork}.

The way to achieve this is by requiring upgraded miners to include the
interlink data structure in the form of a new Merkle tree root hash in their
coinbase data, similar to a soft fork. An unupgraded miner will as usual ignore
this data as comments. However, we further require the upgraded miners to
accept all previously accepted blocks, regardless of whether they have included
the interlink data structure or not. In fact, even if the interlink data
structure is included and contains invalid data, we require the upgraded miners
to accept their containing blocks. Malformed interlink data could be simply of
the wrong format, for example pointers of particular level appearing in the wrong
position in the Merkle tree, or the pointers could be pointing to superblocks
of incorrect levels.  Furthermore, the pointers could be pointing to
superblocks of the correct level, but not to the most recent block. By
requiring upgraded miners to accept all such blocks, we do not modify the set
of accepted blocks: Upgraded and unupgraded miners accept the exact same set of
blocks. Therefore, the upgrade is simply a ``recommendation'' for miners and
not an actual change in the consensus rules. Hence, while a hard fork makes new
upgraded blocks invalid to unupgraded clients and a soft fork makes new
unupgraded blocks invalid to upgraded clients, the velvet fork has the effect
that blocks produced by either upgraded or unupgraded clients are valid for
either. Hence, the blockchain is never forked. Only the codebase is upgraded,
and the existing data on the blockchain is interpreted differently.

The reason this can work is because provers and verifiers of our protocol can
check the validity of the claims of miners who make false interlink chain
claims. An upgraded prover can check whether a block contains correct interlink
data and use it. If a block does not contain correct interlink data, the prover
can opt not to use those pointers at all in their proofs. As the Verifier
verifies all claims of the prover, adversarial miners cannot cause harm by
including invalid interlink data. The only thing that the Verifier cannot
verify in terms of interlink claims is whether the claimed superblock of a
given level is in fact the most recent previous superblock of that level.
However, an adversarial prover cannot make use of that to construct winning
proofs, as they are only able to present shorter chains in that case. The
honest prover can simply ignore such pointers as if they were not included at
all.

The velvet prover works as usual, but additionally maintains a
\textit{realLink} data structure as part of its full node, which stores the
correct interlink for each block.  Whenever a new winning chain is received
from the network, the prover checks it for blocks that it hasn't seen before.
For those blocks, it maintains its own realLink data structure which it updates
accordingly to make sure it is correct regardless of what the interlink data
structure of the received block claims.

\import{./}{algorithms/alg.nipopow-velvet-innerchain.tex}
\import{./}{algorithms/alg.nipopow-velvet-follow.tex}
\import{./}{algorithms/alg.nipopow-velvet-prover.tex}

\paragraph{Supporting clients with different beliefs.}
We note that the interlink format does not depend at all on the parameters $m$ and $k$.
Therefore, regardless of the upgrade procedure, it is not necessary to reach consensus agreement on a particular choice of these parameters. Instead, the choice of $m$ and $k$ can be exposed as a user-configurable parameter to clients.
% Clients would send a particular $m$ and $k$ as a .

