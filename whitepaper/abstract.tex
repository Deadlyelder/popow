Blockchain protocols such as bitcoin provide decentralized consensus mechanisms
based on proof-of-work (PoW). In this work we introduce and instantiate a new
primitive for blockchain protocols called Noninteractive-Proofs-of-Proof-of-Work
(NIPoPoWs) which can be adapted into existing proof-of-work-based
cryptocurrencies. Unlike a traditional blockchain client which must verify the
entire linearly-growing chain of PoWs, clients based on NIPoPoWs require
resources only logarithmic in the length of the blockchain. NIPoPoWs solve two
important open questions for PoW based consensus protocols. Specifically the
problem of constructing efficient transaction verification clients, sometimes
called ``simple payment verification'' or SPV, and the problem of constructing
efficient sidechain proofs. We provide a formal model for NIPoPoWs. We prove our
construction is secure in the backbone model and show that the communication is
succinct in size. We provide simulations and experimental data to measure
concrete communication efficiency and security. Finally, we provide two ways
that our NIPoPoWs can be adopted by existing blockchain protocols, first via a
soft fork, and second via a new update mechanism that we term a ``velvet fork''
and enables to harness some of the performance benefits of NIPoPoWs even with a
minority upgrade.
