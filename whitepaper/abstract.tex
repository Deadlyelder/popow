\begin{abstract}
Blockchain protocols such as bitcoin have created decentralized consensus
mechanisms that allow monetary and other applications. In order to experiment
and propose different consensus rules and avoid hard forks, sidechains were
proposed. To transfer a coin between sidechains, a transaction must include a
proof that proof-of-work has been performed. This proof is linear in size in
the case of both a full node or an SPV. Using lite proofs-of-proofs-of-work,
these proofs can become sublinear, but require interactivity. We modify PoPoWs
to produce non-interactive proofs. We describe how exactly these proofs can be
constructed. We show that these proofs remain sublinear in size. We then prove,
using a classical cryptographic reduction argument, that these proofs provide
equivalent security to interactive proofs. Finally, we argue that their
adoption can be achieved in bitcoin using a soft fork which requires only a
minority of miners to update. This construction provides a solution to the
previously open problem of building secure and efficient sidechains.
\end{abstract}
