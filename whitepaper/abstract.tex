% \begin{abstract}
Blockchain protocols such as bitcoin provide decentralized consensus
mechanisms based on proofs of work (PoWs) 
that allow monetary and other applications. 
%
In this work we introduce and implement a new primitive for blockchain protocols, called Noninteractive-Proofs-of-Proof-of-Work, (NiPoPoWs), which can be adapted into existing cryptocurrencies to support more efficient clients.
Unlike a traditional blockchain client, which must verify the entire chain of PoWs, which grows linearly over time, clients based on NiPoPoWs require resources only logarithmic in the length of the blockchain.

%This proof is linear in size in
%the case of both a full node or an SPV. Using lite proofs-of-proofs-of-work,
%these proofs can become sublinear, but require interactivity. We modify PoPoWs
%to produce non-interactive proofs. We describe how exactly these proofs can be
%constructed. 
NiPoPoWs solve two important open questions for PoW based consensus
protocols. Specifically the problem  
of constructing efficient transaction verification (sometimes called
``simple payment verification'' - SPV) clients and 
the problem of constructing efficient sidechain proofs. 
%
%Sidechains is a mechanism that enables 
% experimenting with different consensus rules while avoid hard forks. 
%Secure implementation to transfer a coin between sidechains, a transaction must include a
%proof that proof-of-work has been performed. t
%
We provide a formal model of NiPoPoWs, a construction that we prove
secure in our model 
via a cryptographic reduction argument and 
an implementation that we benchmark to demonstrate  its efficiency. 
Finally, we provide two ways that our NiPoPoWs can be adopted by
 existing blockchain protocols, first via a soft fork, that enables the 
   adoption of the NiPoPoW related data structures
  as part of the underlying consensus mechanism, 
 and second  via a new update mechanism 
that we term a ``velvet fork'' and enables to harness some of
the performance benefits of NiPoPoWs even without reaching consensus about its use
across nodes. The velvet fork scheme is of independent interest to enable
consensus layer upgrades without requiring an agreement by miner majority.

%This construction provides a solution to the
%previously open problem of building secure and efficient sidechains. Our
%construction is of general interest and can be applied to any PoW-based
%cryptocurrency.
% \end{abstract}

