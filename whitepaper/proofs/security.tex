\begin{proof}
    We will prove security by contradiction.  Let $m = k_1 + k_2$ and let $k =
    k(\kappa)$, $k_1 = k_1(\kappa)$ and $k_2 = k_2(\kappa)$ be polynomial
    functions of the security parameter $\kappa$.  Let $Q$ be a $k$-stable
    suffix sensitive chain predicate. Assume the NIPoPoWs construction with
    parameters $m, k$ is not secure with respect to $Q$. Then, during an
    execution at some round $r_3$, $Q(\chain)$ is defined and has the same
    value for all honest miners.  Assume the execution is typical. The verifier
    $V$ communicates with at least two provers, $\mathcal{A}$, the adversary,
    and $B$, an honest prover.  The verifier receives the proofs
    $\overline{\Pi}_\mathcal{A}, \overline{\Pi}_B$ from the adversary and the
    honest prover respectively.  Because $B$ is honest, $\overline{\Pi}_B$ will
    be a proof constructed based on an underlying blockchain $\chain_B$ which
    is the adopted blockchain of $B$ during round $r_3$. The verifier then
    outputs $\lnot Q(\chain_B)$, and so $\textsf{Verify}^Q_{m,k} = \lnot Q(\chain_B)$.
    Thus it is
    necessary that $\overline{\Pi}_\mathcal{A} \geq
    \overline{\Pi}_B$, otherwise, because $Q$ is suffix
    sensitive, $\textsf{Verify}^Q$ would have returned $Q(\chain_B)$. We will
    now show that $\overline{\Pi}_\mathcal{A} \geq
    \overline{\Pi}_B$ is a negligible event.

    Let $b$ be the LCA of $\overline{\Pi}_\mathcal{A}$ and $\overline{\Pi}_B$
    and let the levels of comparison be $\mu_\mathcal{A}$ and $\mu_B$
    respectively.

    There are
    two ways in which $V$'s call to $\textsf{maxChain}$ can return true. Either
    it happens within the \textit{for} loop and so the comparison is at the
    superchain level, or it happens at the end and so the comparison is at
    level $0$. We will deal with these cases separately.

    \textbf{Case 1:} The $\textsf{maxChain}$ algorithm returned true at level
    $0$.

    This means that the \textit{for} loop was executed without the \textit{if}
    statement ever becoming true, and so $b$ is a block towards the end of
    $\chain_B$. Specifically, $\overline{\Pi}_B$ will contain all the blocks in
    $\chain_B$ after $b$. In order to cause $V$ to output a different value for
    the predicate $Q$, the adversary must present at least $k$ blocks after $b$
    which are not in $\chain_B$. This is because the predicate is $k$-stable.
    But this violates the Common Prefix property and thus occurs with
    negligible probability.

    \textbf{Case 2:} The $\textsf{maxChain}$ algorithm returned true at some
    level $\mu > 0$.

    Let $\chain'_\mathcal{A}$ be the adversarial $\mu$-level superchain after
    block $b$ and $\chain'_B$ be the honest party's $\mu$-level superchain
    after block $b$.

    We will proceed in the proof by showing three successive claims: First,
    $\chain'_\mathcal{A}$ and $\chain'_B$ are disjoint. Second,
    $\chain'_\mathcal{A}$ contains mostly adversarially-generated blocks. And
    third, the adversary is able to produce this $\chain'_\mathcal{A}$ with
    negligible probability.

    \textbf{Claim 1: } $\chain'_\mathcal{A}$ and $\chain'_B$ are dijoint.

    $\mu$ will be the highest level at which $\mathcal{A}$ is able to present $m$
    blocks of level $\mu$ after $b$. Because $\mathcal{A}$ is winning,
    therefore $\mu$ is the highest level at which any of the parties was able
    to present at least $m$ blocks of that level after $b$.  This means that at
    every higher level $\mu + i$ for $i > 0$, it was true that $b$ was within
    the last $m$ blocks of that level.  Therefore, $B$, due to the way the
    proofs are constructed, would have included all the blocks following $b$ of
    that level. Hence, all the $\mu$-level superblocks following $b$ in
    $\chain_B$ have been included in $\overline{\Pi}_B$. Because $b$ is the
    LCA, it follows that $\chain'_\mathcal{A}$ and $\chain'_B$ are disjoint.

    \textbf{Claim 2: } At least $k_2$ superblocks of $\chain'_\mathcal{A}$ are
    adversarially generated.

    We will show this by showing that $\chain'_\mathcal{A}[k_1 + 1:]$ contains
    no honestly mined blocks. By contradiction, assume that the block
    $\chain'_\mathcal{A}[i]$ for some $i \geq k_1 + 1$ was honestly generated.
    But this means that an honest party had adopted the chain
    $\chain'_\mathcal{A}[i - 1]$ at some round $r_2 \leq r_3$. Because of the
    way the honest parties adopt chains, this means that the superchain
    $\chain'_\mathcal{A}[:i - 1]$ has an underlying properly constructed
    $0$-level chain $\chain_\mathcal{A}$ which starts at genesis and includes
    all the blocks of the superchain $\chain'_\mathcal{A}[:i - 1]$. Let $j$ be
    the index of block $b$ within $\chain_\mathcal{A}$.  As
    $\chain_\mathcal{A}$ contains at least all the blocks of the superchain
    $\chain'_\mathcal{A}$, observe that $|\chain_\mathcal{A}[j + 1:]| > i -
    1 \geq k_1$.  Therefore $\chain_\mathcal{A}[:-k_1] \not\preccurlyeq
    \chain_B$.  But, as $\chain_\mathcal{A}$ was adopted by an honest party at
    round $r_2$ which is prior to round $r_3$ during which $\chain_B$ was
    adopted by an honest party, this contradicts the Common Prefix property.
    It follows that with overwhelming probability in $k_1$, the $k_2 = m - k_1$
    last blocks of the adversarial proof have been adversarially mined.

    \textbf{Claim 3: } $\mathcal{A}$ is able to produce $\chain'_\mathcal{A}$
    with negligible probability.

    Let $b'$ be the latest honestly generated block in $\chain'_\mathcal{A}$,
    or $b$ if no honest block exists in $\chain'_\mathcal{A}$. Let $r_1$ be
    the round during which $b'$ was generated. Let $j$ be the (zero-based)
    index of block $b'$ within $\chain'_\mathcal{A}$, or $-1$ if $b' = b$. Let
    $c$ be the number of $\mu$-level superblocks in $\chain_B$ prior to $b$.
    At the end of round $r_1$, $b'$ was diffused and hence all honest party
    chains must all have adopted a chain of at least $c + j + 1$ superblocks of
    level $\mu$.

    Consider now the set $S$ of consecutive rounds starting at round $r_1$ and
    ending at round $r_3$. Let $z$ be the number of $\mu$-level superblocks
    generated by the adversary during those rounds, and observe that $z \geq m
    - (j + 1) \geq k_2$. From the assumption that the execution is typical, it
    follows that either $Z^\mu(S) < k_2$, or that $Z^\mu(S)$ and $X^\mu(S)$
    must be distributed close to their respective means. But in order for the
    adversary to have generated $z$ blocks in $S$, it must hold that $Z^\mu(S)
    \geq k_2$, and therefore $Z^\mu(S) < (1 - \frac{\delta}{2})X^\mu(S)$. However,
    this means that the honest $\mu$-level superchain has grown by more than
    $z$ superblocks and now contains at least $c + m$ superblocks. Therefore,
    the honest chain now contains more than $c + j + 1 + z$ superblocks, while
    the adversarial chain contains exactly $c + j + 1 + z$ superblocks.
    $\textsf{maxChain}$ could not have returned true and
    $\overline{\Pi}_\mathcal{A}$ could not have been the winning proof.
\end{proof}
