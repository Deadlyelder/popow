\begin{figure}[t]
\begin{algorithm}[H]
    \caption{\label{alg.backbone-prover} Our generic honest Prover entity addition to the
        backbone model. It is augmented with a{ \em
        proof generating function} $\textsf{Prove}_Q(\cdot)$ parameterized by a
        predicate $Q$}
    \begin{algorithmic}[1]
     \Statex
     \Let\chain\varepsilon
        \While{\textsc{True}}
            \Let{\tilde\chain}{\mathsf{maxvalid}( \chain, \mbox{any chain $\chain'$ found in  \textsc{Receive}()}) }
            \If{$\textsc{Input}()\mbox{ contains }\textsc{ReadProof}$}
                \Let{\overline \Pi}{\textsf{Prove}_Q(\chain)}
                \State\textsc{Diffuse}{$(\overline \Pi)$}
            \Else
                \State\textsc{Diffuse}{$(\bot)$}
            \EndIf
        \EndWhile
        \vskip8pt
    \end{algorithmic}
\end{algorithm}
\end{figure}
