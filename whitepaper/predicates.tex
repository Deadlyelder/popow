\subsection{Provable chain predicates}

Not all predicates can be proved. In this section we characterize the class of
blockchain predicates that will be of interest to us for constructing proofs.
A useful property of predicates in our context is \textit{monotonicity}. Based
on predicate monotonicity, we show that a predicate has \textit{liveness} and
\textit{persistence}. Formalizing the security of our  blockchain proofs  will
be preconditioned on such blockchain predicates. Given that we are in a
decentralized setting it can be the case that for certain sensible blockchain
predicate, there will be honest nodes that have not reached a conclusion about
its value. For this reason we  allow our predicates to have an undefined value
$\bot$ before they take on a truth value of $0$ or $1$. We define a number of
predicate properties and prove a simple but useful relation between them.

\begin{definition}{(Predicate monotonicity)}
    A chain predicate $Q(\chain)$ is $\textit{monotonic}$ if for all chains
    $\chain$ and for all blocks $B$ we have that:
%\begin{equation*}
    $    Q(\chain) \neq \bot \Rightarrow Q(\chain) = Q(\chain B) $.
%\end{equation*}

{(Predicate switching position)}
    A chain predicate $Q(\chain) \neq \bot$ has a switching position $i \in
    \mathbb{N}$ in the chain $\chain$ defined as the minimum $i$ such that:
%    \begin{equation*}
 $       Q(\chain[:i]) \neq \bot$.
%    \end{equation*}

 {(Predicate stability)}
    Parameterized by $k \in \mathbb{N}$ and a chain $\chain$, a chain predicate
    $Q$ has $\textit{stability}$ if its switching position $i$ satisfies $i <
    |\chain| - k$.

{(Predicate persistence)}
    Parameterized by $k \in \mathbb{N}$ and a chain $\chain$, a chain predicate
    $Q$ has $\textit{persistence}$ if the following is true: When in a certain
    round the predicate is $k$-stable in some honest party's chain, then
    whenever it is $k$-stable in any honest party's chain, the truth value of
    the predicate is the same. Furthermore, the change of truth value from
    $\bot$ to its final truth value happened at the same blockchain depth.

\ignore{%%%%%
{(Predicate liveness)}
    Parameterized by $u, k \in \mathbb{N}$ (the ``wait time'' and ``depth''
    parameters respectively) and a chain $\chain$, a chain predicate
    $Q$ has $\textit{liveness}$ if it is $k$-stable for one honest party
    in a certain round, then it will also be $k$-stable for all honest parties
    after $u$ rounds.
}%%%%
\end{definition}

\begin{theorem}
    If a chain predicate is monotonic, then it satisfies persistence with
    parameter $k = 2\eta \kappa f$ with probability at least $1 -
    e^{-\Omega(\kappa)}$.
\end{theorem}

\begin{proof}
    Assume a typical execution as defined in \cite{backbone} and let $\chain_1$
    be the chain of some honest player $P_1$ at round $r_1$. Assume that the
    chain predicate $Q(\chain)$ is $k$-stable for $P_1$ at round $r_1$, i.e.
    $Q(\chain[:-k]) \neq \bot$.  The chain of any honest player $P_2$ will be
    at least as long from the next round on and from common prefix we have that
    $\chain_1[:-k] \preccurlyeq \chain_2$. Therefore the predicate will be
    $k$-stable, have the same truth value, and have the same switching position
    for all honest parties.
\end{proof}

\ignore{%%%%
\begin{theorem}
    If a chain predicate is monotonic, then it satisfies liveness with
    parameters $u = 1$ rounds and $k = 2\eta \kappa f$ with probability at
    least $1 - e^{-\Omega(\kappa)}$.
\end{theorem}

\begin{proof}
    ?
\end{proof}
}%%%%
